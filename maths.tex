\documentclass{article}

\usepackage{times}
\usepackage[left=1in,right=1in,top=1in,bottom=1in]{geometry}
\usepackage{enumitem}
\usepackage{amsmath}
\usepackage{amssymb}
\usepackage{tikz}
\usepackage{xcolor}

\title{Maths}
\author{Alexander}
\date{\today}

\begin{document}
\maketitle

\section* {Algebra}
	\subsection*{Lines}
		Slope of the line through $P_1 = (x_1, y_1)$ and $P_2 = (x_2, y_2)$:
			\begin{center}
			$m = \frac{y_2 - y_1}{x_2 - x_1}$
			\end{center}
		Slope-intercept equation of line with slope m and y-intercept b:
			\begin{center}
			$y = mx + b$
			\end{center}
		Point-slope equation of line through $P_1 = (x_1, y_1)$ with slope m:
			\begin{center}
			$y - y_1 = m(x - x_1)$
			\end{center}
	\subsection*{Circles}
		Equation of the circle with center (a, b) and radius r:
			\begin{center}
			$(x - a)^2 + (y - b)^2 = r^2$
			\end{center}
	\subsection*{Distance and Midpoint Formulas}
		Distance between $P_1 = (x_1, y_1)$ and $P_2 = (x_2, y_2)$:
			\begin{center}
			$d = \sqrt{(x_2 - x_1)^2 + (y_2 - y_1)^2}$
			\end{center}
		Midpoint of $\bar{P_1P_2}$:
			\begin{center}
			$(\frac{x_1 + x_2}{2}, \frac{y_1 + y_2}{2})$
			\end{center}
	\subsection*{Laws of Exponents}
		\begin{center}
		$x^m x^n = x^{m+n}$\\
		\vspace{10pt}
		$\frac{x^m}{x^n} = x^{m-n}$\\
		\vspace{10pt}
		$(x^m)^n = x^{mn}$\\
		\vspace{10pt}
		$x^{-n} = \frac{1}{x^n}$\\
		\vspace{10pt}
		$(xy)^n = x^n y^n$\\
		\vspace{10pt}	
		$(\frac{x}{y})^n = \frac{x^n}{y^n}$\\	
		\vspace{10pt}	
		$x^{\frac{1}{n}} = \sqrt[n]{x}$\\	
		\vspace{10pt}	
		$\sqrt[n]{xy} = \sqrt[n]{x}\sqrt[n]{y}$\\
		\vspace{10pt}
		$\sqrt[n]{\frac{x}{y}} = \frac{\sqrt[n]{x}}{\sqrt[n]{y}}$\\
		\vspace{10pt}
		$x^{\frac{m}{n}} = \sqrt[n]{x^m} = (\sqrt[n]{x})^m$
		\end{center}
	\subsection*{Special Factorizations}
		\begin{center}
		$x^2 - y^2 = (x + y)(x - y)$\\
		\vspace{10pt}
		$x^3 + y^3 = (x + y)(x^2 - xy + y^2)$\\
		\vspace{10pt}
		$x^3 - y^3 = (x-y)(x^2 + xy + y^2)$
		\end{center}
	\subsection*{Binomial Theorem}
		\begin{center}
		$(x+y)^2 = x^2 + 2xy +y^2$\\
		\vspace{10pt}
		$(x-y)^2 = x^2 - 2xy + y^2$\\
		\vspace{10pt}
		$(x+y)^3 = x^3 + 3x^2y + 3xy^2 + y^3$\\
		\vspace{10pt}
		$(x-y)^3 = x^3 - 3x^2y + 3xy^2 - y^3$\\
		\vspace{10pt}
		$(x+y)^n = x^n + nx^{n-1}y + \frac{n(n-1)}{2}x^{n-2}y^2 + \ldots + \binom{n}{k}x^{n-k}y^k + \ldots + nxy^{n-1} + y^n$\\
		where $\binom{n}{k} = \frac{n(n-1)\ldots(n-k+1)}{1 \cdot 2 \cdot 3 \ldots k}$
		\end{center}
	\subsection*{Quadratic Formula}
		\begin{center}
		If $ax^2 + bx +c = 0$, then $x = \frac{-b \pm \sqrt{b^2 - 4ac}}{2a}$.
		\end{center}
	\subsection*{Inequalities and Absolute Value}
		\begin{center}
		If $a < b$ and $b < c$, then $a < c$.\\
		\vspace{10pt}
		If $a < b$, then $a + c < b + c$.\\
		\vspace{10pt}
		if $a < b$ and $c > 0$, then $ca < cb$.\\
		\vspace{10pt}
		if $a < b$ and $c < 0$, then $ca > cb$.\\
		\vspace{10pt}
		$\left|x\right| = x$ if $x >= 0$\\
		\vspace{10pt}
		$\left|x\right| = -x$ if $x <= 0$
		\end{center}
		
\section* {Geometry}
	Formulas for area A, circumference C, and volume V\\
	Triangle\\
		\begin{center}
		$A = \frac{1}{2}bh$\\
		\vspace{10pt}
		$A = \frac{1}{2}ab\sin(\theta)$\\
		\end{center}
	Circle\\
		\begin{center}
		$A = \pi r^2$\\
		\vspace{10pt}
		$C = 2\pi r$\\
		\end{center}
	Sector of Circle\\
		\begin{center}
		$A = \frac{1}{2}r^2\theta$\\
		\vspace{10pt}
		$s = r\theta$\\
		\end{center}
	Sphere\\
		\begin{center}
		$V = \frac{4}{3}\pi r^3$\\
		\vspace{10pt}
		$A = 4\pi r^2$\\
		\end{center}
	Cylinder\\
		\begin{center}
		$V = \pi r^2h$\\
		\end{center}
	Cone\\
		\begin{center}
		$V = \frac{1}{3}\pi r^2h$\\
		\vspace{10pt}
		$A = \pi r\sqrt{r^2 + h^2}$\\
		\end{center}
	Cone with arbitrary base\\
		\begin{center}
		$V = \frac{1}{3}Ah$\\
		\end{center}
		
\section* {Trigonometry}
	Pythagorean Theorem: For a right trianlge with hypotenuse of length $c$ and legs of lengths $a$ and $b$, $c^2 = a^2 + b^2$.\\
	\subsection*{Angle Measurement}
		\begin{center}
		$\pi$ radians = $180^\circ$\\
		\vspace{10pt}
		$1^\circ = \frac{\pi}{180}rad$\\
		\vspace{10pt}
		1 rad = $\frac{180}{\pi}$\\
		\vspace{10pt}
		$s = r\theta$ ($\theta$ in radians)
		\end{center}
	\subsection*{Right Triangle Definitions}
		\begin{center}
		sin $\theta = \frac{opp}{hyp}$\\
		\vspace{10pt}
		cos $\theta = \frac{adj}{hyp}$\\
		\vspace{10pt}
		tan $\theta = \frac{sin \theta}{cos \theta} = \frac{opp}{adj}$\\
		\vspace{10pt}
		sec $\theta = \frac{1}{cos \theta}$ 
		\vspace{10pt}
		csc $\theta = \frac{1}{sin \theta}$
		\end{center}
	\subsection*{Trigonometric Functions}
		\begin{center}
		sin $\theta = \frac{y}{r}$\\
		\vspace{10pt}
		cos $\theta = \frac{x}{r}$\\
		\vspace{10pt}
		tan $\theta = \frac{y}{x}$\\
		\vspace{10pt}
		sec $\theta = \frac{r}{x}$\\
		\vspace{10pt}
		csc $\theta = \frac{r}{y}$\\
		\vspace{10pt}
		$\lim_{\theta \to 0} \frac{sin \theta}{\theta} = 1$\\
		\vspace{10pt}
		$\lim_{\theta \to 0} \frac{1 - cos \theta}{\theta} = 0$
		\end{center}
	\subsection*{Fundamental Identities}
		\begin{center}
		$sin^2 \theta + cos^2 \theta = 1$\\
		\vspace{10pt}
		$1 + tan^2 \theta = sec^2\theta$\\
		\vspace{10pt}
		$1 + cot^2 \theta = csc^2 \theta$\\
		\vspace{10pt}
		$\sin(\frac{\pi}{2} - \theta) = \cos(\theta)$\\
		\vspace{10pt}
		$\cos(\frac{\pi}{2} - \theta) = \sin(\theta)$\\
		\vspace{10pt}
		$\tan(\frac{\pi}{2} - \theta) = \cot(\theta)$\\
		\vspace{10pt}
		$\sin(-\theta) = -sin\theta$\\
		\vspace{10pt}
		$\cos(-\theta) = cos\theta$\\
		\vspace{10pt}
		$\tan(-\theta) = -tan\theta$\\
		\vspace{10pt}
		$\sin(\theta + 2\pi) = sin\theta$\\
		\vspace{10pt}
		$\cos(\theta + 2\pi) = cos\theta$\\
		\vspace{10pt}
		$\tan(\theta + \pi) = tan\theta$
		\end{center}
	\subsection*{The Law of Sines}
		\begin{center}
		$\frac{sin A}{a} = \frac{sin B}{b} = \frac{sin C}{c}$
		\end{center}
	\subsection*{The Law of Cosines}
		\begin{center}
		$a^2 = b^2 + c^2 - 2bc cos A$
		\end{center}
	\subsection*{Addition and Subtraction Formulas}
		\begin{center}
		$\sin(x + y) = sinxcosy + cosxsiny$\\
		\vspace{10pt}
		$\sin(x - y) = sinxcosy - cosxsiny$\\
		\vspace{10pt}
		$\cos(x + y) = cosxcosy - sinxsiny$\\
		\vspace{10pt}
		$\cos(x - y) = cosxcosy + sinxsiny$\\
		\vspace{10pt}
		$tan(x + y) = \frac{tanx + tany}{1 - tanxtany}$\\
		\vspace{10pt}
		$tan(x - y) = \frac{tanx - tany}{1 + tanxtany}$
		\end{center}
	\subsection*{Double-Angle Formulas}
		\begin{center}
		$sin2x = 2sinxcosx$\\
		\vspace{10pt}
		$cos2x = cos^2x - sin^2x = 2cos^2x-1 = 1-2sin^2x$\\
		\vspace{10pt}
		$tan2x = \frac{2tanx}{1 - tan^2x}$\\
		\vspace{10pt}
		$sin^2x = \frac{1 - cos2x}{2}$\\
		\vspace{10pt}
		$cos^2x = \frac{1+cos2x}{2}$
		\end{center}

\section* {Precalculus Review}

78. Prove the triangle inequality by adding the two inequalities\\

1.) Known Inequalities\\
\indent$-\left|a\right| \leq a \leq \left|a\right|$\\
\indent$-\left|b\right| \leq b \leq \left|b\right|$\\

2.) Add and Simplify\\
\indent$(-\left|a\right|)+(-\left| b\right|) \leq a + b \leq \left|a\right| + \left|b\right|$\\
\indent$-(\left|a\right| + \left|b\right|) \leq a + b \leq \left|a\right| + \left|b\right|$\\

3.) By the definition of absolute value, we know:\\
\indent$-\left|x\right| \leq x \leq \left|x\right|$\\
\indent$-\left|a+b\right| \leq a+b \leq \left|a+b\right|$\\

4.) To explain, the value $a + b$ is squeezed between $-(\left|a\right| + \left|b\right|)$ and $\left|a\right| + \left|b\right|$. By taking the absolute value on both sides, we conclude that: \\
\indent$-(\left|a\right| + \left|b\right|) \leq a + b \leq \left|a\right| + \left|b\right|$\\
\indent$\left|a + b\right| \leq \left|a\right| + \left|b\right|$\\

\newpage

79. Show that if $r=\frac{a}{b}$ is a fraction in lowest terms, then $r$ has a finite decimal expansion if and only if $b = (2^n)(5^m)$ for some n, m $\geq$ 0. Hint: Observe that r has a finite decimal expansion when $(10^N)(r)$ is an integer for some $N\geq0$ (and hence b dividies $10^N$).\\

1.) Understanding Finite Decimal Expansion\\

A fraction $\frac{a}{b}$ has a finite decimal expansion if and only if, after simplification, the denominator b only contains the prime factos 2 and 5. In other words, $b$ must be of the form $2^n \cdot 5^m$, where $n$ and $m$ are non-negative integers. To see why this is true, recall that a fraction $\frac{a}{b}$ has a finite decimal expansion if and only if $b$ divides $10^N$ for some integer $N$. This is because multiplying $\frac{a}{b}$ by $10^N$ (where $N$ is large enough) should result in an integer.\\

2.) Necessity: $b$ Divides $10^N$\\

Claim: If $\frac{a}{b}$ has a finite decimal expansion, then $b$ must be of the form $2^n \cdot 5^m$.\\

Proof:\\

Assume $\frac{a}{b}$ has a finite decimal expansion.Then there exists some integer $N \geq 0$ such that: $(10^N) \cdot (\frac{a}{b})$ is an integer.\\

Rearranging, we get: $\frac{10^N \cdot a}{b}$ is an integer\\

This implies that $b$must divide $10^N$, since $10^N \cdot a$ is an integer and $b$ is a factor of this integer.\\

The number $10^N$ can be factored into primes as: $10^N = 2^N \cdot 5^N$.\\

Therefore, if $b$ divides $10^N$, $b$ must be of theform $2^n \cdot 5^m$ where $n \leq N$ and $m \leq N$. This is because the only prime factors of $10^N$ are 2 and 5.\\

3.)  Sufficiency: $b$ of the form $2^n \cdot 5^m$\\

Claim: if $b$ can be expressed as $2^n \cdot 5^m$ forsome non-negative integers n andm, then $\frac{a}{b}$ has a finite decimal expansion.\\

Proof:\\

Suppose $b = 2^n \cdot 5^m$. We need to show that $\frac{a}{b}$ has a finite decimal expansion.\\

To have a finite decimal expansion, there must be some integer $N \geq 0$ such that: $10^N = 2^N \cdot 5^N$.\\

Since $10^N$ contains at least $2^n$ and $5^m$ (as $N \geq n$ and $N \geq m$), we have: $10^N = 2^N \cdot 5^N$ is divisible by $b = 2^n \cdot 5^m$.\\

Therefore: $\frac{10^N \cdot a}{b}$ is an integer because $10^N$ contains all the prime factors of $b$\\

Conclusion: \\

1. If $\frac{a}{b}$ has a finite decimal expansion, then $b$ must be of the for $2^n \cdot 5^m$ for non-negative integers $n$ and $m$.\\
2. Conversely, if $b$ is of the form $2^n \cdot 5^m$, then $\frac{a}{b}$ will have a finite decimal expansion\\

Thus, the fraction $\frac{a}{b}$ in lowest terms has a finite decimal expansion if and only if $b = 2^n \cdot 5^m$ for some non-negative integers $n$ and $m$.

\end{document}






























