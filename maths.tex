\documentclass{article}

\usepackage{times}
\usepackage[left=1in,right=1in,top=1in,bottom=1in]{geometry}
\usepackage{enumitem}
\usepackage{amsmath}
\usepackage{amssymb}
\usepackage{tikz}

\title{Maths}
\author{Alexander}
\date{\today}

\begin{document}
\maketitle

\section* {Algebra}
	\subsection*{Lines}
		Slope of the line through $P_1 = (x_1, y_1)$ and $P_2 = (x_2, y_2)$:
			\begin{center}
			$m = \frac{y_2 - y_1}{x_2 - x_1}$
			\end{center}
		Slope-intercept equation of line with slope m and y-intercept b:
			\begin{center}
			$y = mx + b$
			\end{center}
		Point-slope equation of line through $P_1 = (x_1, y_1)$ with slope m:
			\begin{center}
			$y - y_1 = m(x - x_1)$
			\end{center}
	\subsection*{Circles}
		Equation of the circle with center (a, b) and radius r:
			\begin{center}
			$(x - a)^2 + (y - b)^2 = r^2$
			\end{center}
	\subsection*{Distance and Midpoint Formulas}
		Distance between $P_1 = (x_1, y_1)$ and $P_2 = (x_2, y_2)$:
			\begin{center}
			$d = \sqrt{(x_2 - x_1)^2 + (y_2 - y_1)^2}$
			\end{center}
		Midpoint of $\bar{P_1P_2}$:
			\begin{center}
			$(\frac{x_1 + x_2}{2}, \frac{y_1 + y_2}{2})$
			\end{center}
	\subsection*{Laws of Exponents}
		\begin{center}
		$x^m x^n = x^{m+n}$\\
		
		$\frac{x^m}{x^n} = x^{m-n}$\\
		
		$(x^m)^n = x^{mn}$\\
		
		$x^{-n} = \frac{1}{x^n}$\\
		
		$(xy)^n = x^n y^n$\\
		
		$(\frac{x}{y})^n = \frac{x^n}{y^n}$\\
		
		$x^{\frac{1}{n}} = \sqrt[n]{x}$\\
		
		$\sqrt[n]{xy} = \sqrt[n]{x}\sqrt[n]{y}$\\
		
		$\sqrt[n]{\frac{x}{y}} = \frac{\sqrt[n]{x}}{\sqrt[n]{y}}$\\
		
		$x^{\frac{m}{n}} = \sqrt[n]{x^m} = (\sqrt[n]{x})^m$
		\end{center}
	\subsection*{Special Factorizations}
		\begin{center}
		$x^2 - y^2 = (x + y)(x - y)$\\
		$x^3 + y^3 = (x + y)(x^2 - xy + y^2)$\\
		$x^3 - y^3 = (x-y)(x^2 + xy + y^2)$
		\end{center}
	\subsection*{Binomial Theorem}
		\begin{center}
		$(x+y)^2 = x^2 + 2xy +y^2$\\
		$(x-y)^2 = x^2 - 2xy + y^2$\\
		$(x+y)^3 = x^3 + 3x^2y + 3xy^2 + y^3$\\
		$(x-y)^3 = x^3 - 3x^2y + 3xy^2 - y^3$
		$(x+y)^n = x^n + nx^{n-1}y + \frac{n(n-1)}{2}x^{n-2}y^2 + \ldots + \binom{n}{k}x^{n-k}y^k + \ldots + nxy^{n-1} + y^n$\\
		where $\binom{n}{k} = \frac{n(n-1)\ldots(n-k+1)}{1 \cdot 2 \cdot 3 \ldots k}$
		\end{center}
	\subsection*{Quadratic Formula}
		\begin{center}
		If $ax^2 + bx +c = 0$, then $x = \frac{-b \pm \sqrt{b^2 - 4ac}}{2a}$.
		\end{center}
	\subsection*{Inequalities and Absolute Value}
		\begin{center}
		If $a < b$ and $b < c$, then $a < c$.\\
		If $a < b$, then $a + c < b + c$.\\
		if $a < b$ and $c > 0$, then $ca < cb$.\\
		if $a < b$ and $c < 0$, then $ca > cb$.\\
		$\left|x\right| = x$ if $x >= 0$\\
		$\left|x\right| = -x$ if $x <= 0$
		\end{center}
		
\section* {Geometry}
	Formulas for area A, circumference C, and volume V\\
	Triangle\\
		\begin{center}
		$A = \frac{1}{2}bh$\\
		$A = \frac{1}{2}ab\sin(\theta)$\\
		\end{center}
	Circle\\
		\begin{center}
		$A = \pi r^2$\\
		$C = 2\pi r$\\
		\end{center}
	Sector of Circle\\
		\begin{center}
		$A = \frac{1}{2}r^2\theta$\\
		$s = r\theta$\\
		\end{center}
	Sphere\\
		\begin{center}
		$V = \frac{4}{3}\pi r^3$\\
		$A = 4\pi r^2$\\
		\end{center}
	Cylinder\\
		\begin{center}
		$V = \pi r^2h$\\
		\end{center}
	Cone\\
		\begin{center}
		$V = \frac{1}{3}\pi r^2h$\\
		$A = \pi r\sqrt{r^2 + h^2}$\\
		\end{center}
	Cone with arbitrary base\\
		\begin{center}
		$V = \frac{1}{3}Ah$\\
		\end{center}
		
\section* {Trigonometry}
	Pythagorean Theorem: For a right trianlge with hypotenuse of length $c$ and legs of lengths $a$ and $b$, $c^2 = a^2 + b^2$.\\
	\subsection*{Angle Measurement}
	\subsection*{}
	\subsection*{}
	\subsection*{}
	\subsection*{}
	\subsection*{}
	\subsection*{}
	\subsection*{}
\section* {Precalculus Review}


\end{document}
