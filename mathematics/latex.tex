\documentclass{article} 
\usepackage{amsmath}
\usepackage{graphicx}
\usepackage[a4paper, left=1.5in, right=1.5in, top=1.5in, bottom=1.5in]{geometry}
\usepackage{xcolor}

\begin{document}

\title{calculus is a branch of mathematics that deals with rates of change and the accumulation of quantities.}

\maketitle


circle: $(x-a)^2 + (y-b)^2 = r^2$ where $(a, b)$ is the center and the radius is $r$\\

midpoint between $P_1 = (x_1, y_1)$ and $P_2 = (x_2, y_2)$ is $\bar{P_{1}P_2}$ divided by 2\\

distance: $\sqrt{(x_2 - x_1)^2 + (y_2 - y_1)^2}$\\

laws of exponents: 

	\begin{itemize}
		\item $x^mx^n = x^{m+n}$
		\item $\frac{x^m}{x^n} = x^{m-n}$
		\item $(x^m)^n = x^{mn}$
		\item $x^{-n} = \frac{1}{x^n}$
		\item $(xy)^n = x^ny^n$
		\item $(\frac{x}{y})^n = \frac{x^n}{y^n}$
		\item $x^{1/n} = \sqrt[n]{x}$
		\item $x^{\frac{m}{n}} = \sqrt[n]{x^m} = (\sqrt[n]{x})^m$
	\end{itemize}

special factorizations

	\begin{itemize}
		\item $x^2 - y^2 = (x + y)(x - y)$
		\item $x^3 + y^3 = (x + y)(x^2 - xy + y^2)$
		\item $x^3 - y^3 = (x - y)(x^2 + xy + y^2)$
	\end{itemize}

binomial theorem:

	\begin{itemize}
		\item $(x + y)^2 = x^2 + 2xy + y^2$
		\item $(x - y)^2 = x^2 - 2xy + y^2$
		\item $(x + y)^3 = x^3 + 3x^2y + 3xy^2 + y^3$
		\item $(x - y)^3 = x^3 - 3x^2y + 3xy^2 - y^3$
		\item $(x + y)^n = x^n + nx^{n-1}y + \frac{n(n-1)}{2}x^{n-2}y^2 + \ldots + \binom{n}{k}x^{n-k}y^k + \dots + nxy^{n-1} + y^n$
			where $\binom{n}{k} = \frac{n(n-1) \dots (n-k+1)}{1 \cdot 2 \cdot 3 \cdot \ldots \cdot k}$
	\end{itemize}

quadratic formulae:

	\begin{enumerate}
		\item $ax^2 + bx + c = 0$
		\item $x^2 + \frac{b}{a}x + \frac{c}{a} = 0$
		\item $x^2 + \frac{b}{a}x = -\frac{c}{a}$
		\item $x^2 + \frac{b}{a}x + \left( \frac{b}{2a} \right)^2 = -\frac{c}{a} + \left( \frac{b}{2a} \right)^2$
		\item $\left( x + \frac{b}{2a} \right)^2 = \frac{b^2 - 4ac}{4a^2}$
		\item $x + \frac{b}{2a} = \pm \frac{\sqrt{b^2 - 4ac}}{2a}$
		\item $x = -\frac{b}{2a} \pm \frac{\sqrt{b^2 - 4ac}}{2a}$
		\item $x = \frac{-b \pm \sqrt{b^2 - 4ac}}{2a}$
	\end{enumerate}

polynomials $a_nx^n + a_{n-1}x^{n-1} + ... + a_1x + a_0$ where n is non-neg and represents the degree\\

rational $p(x)/q(x)$\\

root $\sqrt[n]{g(x)}$\\

properties:\\
	domain: set of all input values for which the function is defined\\
	range: set of all possible output values of the function\\
	
	continuity: most algebaic functions are continuous (no breaks of jumps), but rational functions have discontinuities at points\\
		
	behavior: the functions behavior is influenced bu the degree of the polynomial and the nature of the function\\


	scaling:\\
		\begin{itemize}
			\item{vertical scaling} $y = kf(x)$: If $k \geq 1$, the graph is expanded vertically by the factor $k$. If $0 < k < 1$, the graph is compressed vertically. When the scale factor $k$ is negative $(k < 0)$, the graph is also reflected across the x-axis.
			\item{horizontal scaling} $y = f(kx)$: If $K \geq 1$, the graph is compressed in the horizontal direction. If $0 < k < 1$, the graph is expanded. If $k \leq 0$, then the graph is also reflected across the y-axis.
		\end{itemize}

$1 \deg = \frac{\pi}{180}$rad\\

trigonometric functions are mathematical functions that relate the angle of a trianlge to the lengths of its sides. these functions are based on the relationships between the angles and sides of right-angled triangles and can also be generalized to all real numbers using the unit circle.\\

To derive the rest of the fundamental trigonometric identities, you need a combination of a few key identities and principles. The most important starting point is the Pythagorean identity, but you’ll also need the basic relationships between the trigonometric functions, such as the definitions of sine, cosine, tangent, secant, cosecant, and cotangent in terms of a right triangle or the unit circle. \\

$ \sin^2 \theta + \cos^2 \theta = 1 $\\

$ \sec \theta = \frac{1}{\cos \theta}, \quad \csc \theta = \frac{1}{\sin \theta}, \quad \cot \theta = \frac{1}{\tan \theta} $\\

$ \tan \theta = \frac{\sin \theta}{\cos \theta}, \quad \cot \theta = \frac{\cos \theta}{\sin \theta} $\\

$ 1 + \tan^2 \theta = \sec^2 \theta $\\

$ 1 + \cot^2 \theta = \csc^2 \theta $\\

$ \sin(\alpha + \beta) = \sin \alpha \cos \beta + \cos \alpha \sin \beta $\\

$ \cos(\alpha + \beta) = \cos \alpha \cos \beta - \sin \alpha \sin \beta $\\

$ \tan(\alpha + \beta) = \frac{\tan \alpha + \tan \beta}{1 - \tan \alpha \tan \beta} $\\

$ \sin(\alpha - \beta) = \sin \alpha \cos \beta - \cos \alpha \sin \beta $\\

$ \cos(\alpha - \beta) = \cos \alpha \cos \beta + \sin \alpha \sin \beta $\\

$ \tan(\alpha - \beta) = \frac{\tan \alpha - \tan \beta}{1 + \tan \alpha \tan \beta} $\\

$ \sin(2\theta) = 2\sin \theta \cos \theta $\\

$ \cos(2\theta) = \cos^2 \theta - \sin^2 \theta = 2\cos^2 \theta - 1 = 1 - 2\sin^2 \theta $\\

$ \tan(2\theta) = \frac{2\tan \theta}{1 - \tan^2 \theta} $\\

$ \sin(90^\circ - \theta) = \cos \theta, \quad \cos(90^\circ - \theta) = \sin \theta $\\

$ \tan(90^\circ - \theta) = \cot \theta, \quad \cot(90^\circ - \theta) = \tan \theta $\\

$ \sec(90^\circ - \theta) = \csc \theta, \quad \csc(90^\circ - \theta) = \sec \theta $\\

$ \sin(-\theta) = -\sin(\theta), \quad \cos(-\theta) = \cos(\theta) $\\

$ \tan(-\theta) = -\tan(\theta), \quad \sec(-\theta) = \sec(\theta) $\\

$ \csc(-\theta) = -\csc(\theta), \quad \cot(-\theta) = -\cot(\theta) $\\

$ \sin \alpha \sin \beta = \frac{1}{2} [\cos(\alpha - \beta) - \cos(\alpha + \beta)] $\\

$ \cos \alpha \cos \beta = \frac{1}{2} [\cos(\alpha - \beta) + \cos(\alpha + \beta)] $\\

$ \sin \alpha \cos \beta = \frac{1}{2} [\sin(\alpha + \beta) + \sin(\alpha - \beta)] $\\

power functions: $f(x) = x^n$....if even the function behaves symmetrical around the y-axis...if odd then the function has point symmetry ($x^4$, $x^3$, $x^{-n} = \frac{1}{x^n}$)\\

inverse trig functions: $\arcsin(x) = \sin^{-1}(x) = \theta$, $\arccos(x) = \cos^{-1}(x) = \theta$, $\arcsin(x) = \sin^{-1}(x) = \theta$\\

logs: $\log_ax=y \leftrightarrow a^y = x$, $\ln(x)=y \leftrightarrow e^y=x$\\

hyperbolic functions: $\sinh(x) = \frac{e^x-e^{-x}}{2}$, $\cosh(x) = \frac{e^x+e^{-x}}{2}$, $\tanh(x) = \frac{\sinh(x)}{\cosh(x)}$\\

differentiation rules:
	\begin{enumerate}
		\item$\frac{d}{dx}(c) = 0$
		\item$\frac{d}{dx}x = 1$
		\item$\frac{d}{dx}(x^n) = nx^{-1}$ (power rule)
		\item$\frac{d}{dx}[cf(x)] = cf'(x)$
		\item$\frac{d}{dx}[f(x)+g(x)] = f'(x) + g'(x)$
		\item$\frac{d}{dx}[f(x)g(x)] = f(x)g'(x) + g(x)f'(x)$ (product rule)
		\item$\frac{d}{dx}[\frac{f(x)}{g(x)}] = \frac{g(x)f'(x) - f(x)g'(x)}{[g(x)]^2}$ (quotient rule)
		\item$\frac{d}{dx}$ (chain rule)
		\item$\frac{d}{dx}$ (quotient rule)
		\item$\frac{d}{dx}$ (chain rule)
		\item$\frac{d}{dx}$ (general power rule)
		\item$\frac{d}{dx}$ 
		\item$\frac{d}{dx}$
		\item$\frac{d}{dx}$
		\item$\frac{d}{dx}$
		\item$\frac{d}{dx}$
		\item$\frac{d}{dx}$
		\item$\frac{d}{dx}$
		\item$\frac{d}{dx}$
		\item$\frac{d}{dx}$
		\item$\frac{d}{dx}$
		\item$\frac{d}{dx}$
		\item$\frac{d}{dx}$
		\item$\frac{d}{dx}$
		\item$\frac{d}{dx}$
		\item$\frac{d}{dx}$
		\item$\frac{d}{dx}$
		\item$\frac{d}{dx}$
		\item$\frac{d}{dx}$
		\item$\frac{d}{dx}$	
	\end{enumerate}

\newpage
\section*{problems}	
				
\end{document}























