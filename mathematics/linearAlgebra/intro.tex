\documentclass{article} 
\usepackage{amsmath}
\setlength{\parindent}{0pt}

\usepackage{graphicx}
\usepackage[a4paper, left=1.5in, right=1.5in, top=1.5in, bottom=1.5in]{geometry}
\begin{document}

\title{Linear Algebra}
\author{alexander}
\maketitle

\begin{enumerate}
	\item Linear Equations and Vectors
	\item Matrices and Linear Transformation
	\item Determinates and Eigenvectors
	\item General Vector Spaces
	\item Coordinate Representations
	\item Inner Product Spaces
	\item Numerical Methods
	\item Linear Programming
\end{enumerate}

linear algebra is a branch of mathematics that deals with the study of vector spaces, linear transformations, and systems of linear equations. it provides a framework for describing and analyzing linear relationships between variables\\

	A linear equation in $n$ variables $x_1,x_2,x_3,\ldots,x_n$: $a_1x_1 + a_2x_2 + a_3x_3 + \ldots + a_nx_n = b$ where the coefficients $a_1,a_2,\ldots,a_n$ and b are constants. The following is a system of linear equations: 	
	\begin{align*}
		x_1 + x_2 + x_3 &= 2\\
		2x_1 +3x_2 + x_3 &= 3\\
		x_1 - x_2 - 2x_3 &= -6	
	\end{align*}	

	NOTE TO SELF: 1. three equations... so you are going to have three geometrical objects 2. three variables... those objects are going to be embedded in a three dimensional space ... 3. I cannot necessarily formulate the words here but observe the following...:
		\begin{itemize}
			\item equation: $x = a$, space: 1 dimension(s), object(s) 0 dimension(s)
			\item equation: $ax + by = c$, space: 2 dimension(s), object(s) 1 dimension(s) 
			\item equation: $ax + by + cz = d$, space: 3 dimension(s), object(s) 2 dimension(s) 
			\item equation: $a_1x_1 + \ldots + a_nx_n = d$, space: n dimension(s), object(s) n-1 dimension(s) 
		\end{itemize}

	NOTE TO SELF: remember subsitution? imagine a system with two equations and two unknowns... you solve for one variable in terms of another ... allowing you to have an equation with only one unknown... once a variable is know... you can then use that known variable in an original equation to obtain the remaining unknown variable\\

	As the number of variables increases, a geometrical interpretation os such a system of equations becomes increasingly complex. Each equation will represent a space embedded in a larger space. Solutions will be points that lie on all the embedded spaces.\\

$
\begin{bmatrix}
	1 & 2 & 3 & 61\\
	4 & 5 & 6 & 32\\
	7 & 8 & 9 & 3
\end{bmatrix}
$\\

NOTE TO SELF: size is rows x columns so the size of the matrix above is 3 x 4 and the number 7 is in position row 3 column 1 or (3, 1)

An identity matrix is a square matrix with 1s in the diagonal locations (1,1), (2,2), (3,3), etc., and zeors elsewhere. We write $I_n$ for the n x n identity matrix ... ex.:\\

$
\begin{bmatrix}
	1 & 0 & 0\\
	0 & 1 & 0\\
	0 & 0 & 1
\end{bmatrix}
$\\









\end{document}
