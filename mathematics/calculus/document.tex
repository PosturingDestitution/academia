\documentclass{article}
\usepackage{amsmath}
\usepackage{amssymb}
\setlength{\parindent}{0pt}

\usepackage{graphicx}
\usepackage[a4paper, left=1.5in, right=1.5in, top=1.5in, bottom=1.5in]{geometry}
\usepackage{xcolor}

\usepackage{pgfplots}
\usetikzlibrary{intersections,angles,quotes}
\usepgfplotslibrary{fillbetween}

\title{calculus review document}

\date{\today}

\begin{document}
\maketitle

\section*{introduction}

	A real number ($a \in \mathbb{R}$) is a number represented by a decimal or decimal expansion. There are three types of decimal expansions: finite, repeating, and infinite but nonrepeating. The set of integers is commonly denoted by the letter $\mathbb{Z} = {\dots, -2, -1, 0, 1 , 2, \dots}$. Natural numbers ($\mathbb{N}$) are the set of numbers used for couting and sometimes ordering. A whole number is a non-negative integer. A number is rational ($\mathbb{Q}$) if it can be represented by a fraction $\frac{p}{q}$ where $p$ and $q$ are integers with $q \neq 0$. Rational numbers have a finite or repeating decimal expansions and irrational numbers have infinite, nonrepeating decimal expansions. Furthermore, the decimal expansion of a number is unique, apart from the following exception: Every finite decimal is equal to an infinite decimal in which the digit 9 repeats.\\

	midpoint: $(\frac{x_1 + x_2}{2}, \frac{y_1 + y_2}{2})$\\
	distance: $\sqrt{(x_2 - x_1)^2 + (y_2 - y_1)^2}$\\
	circle: $(x - h)^2 + (y - k)^2 = r^2$\\
	pythagorean: $c^2 = a^2 + b^2$\\

	triangle:
		\begin{itemize}
			\item $A = \frac{1}{2}bh$
			\item $A = \frac{1}{2}ab\sin(\theta)$
		\end{itemize}

	circle:
		\begin{itemize}
			\item $A = \pi r^2$
			\item $C = 2\pi r$
		\end{itemize}

	sector of circle:
		\begin{itemize}
			\item $A = \frac{1}{2}r^2\theta$
			\item $S = r\theta$
		\end{itemize}

	sphere:
		\begin{itemize}
			\item $V = \frac{4}{3}\pi r^3$
			\item $A_s = 4\pi r^2$
		\end{itemize}

	cylinder:
		\begin{itemize}
			\item $V = \pi r^2h$
		\end{itemize}

	cone:
		\begin{itemize}
			\item $V = \frac{1}{3}\pi r^2h$
			\item $A_s = \pi r\sqrt{r^2+h^2}$
		\end{itemize}

	cone with arbitrary base: where A is the area of the base
		\begin{itemize}
			\item $V = \frac{1}{3}Ah$
		\end{itemize}

\textbf{laws of exponents:} 
	\begin{itemize}
		\item $x^mx^n = x^{m+n}$
		\item $\frac{x^m}{x^n} = x^{m-n}$
		\item $(x^m)^n = x^{mn}$
		\item $x^{-n} = \frac{1}{x^n}$
		\item $(xy)^n = x^ny^n$
		\item $(\frac{x}{y})^n = \frac{x^n}{y^n}$ \item $x^{1/n} = \sqrt[n]{x}$
		\item $x^{\frac{m}{n}} = \sqrt[n]{x^m} = (\sqrt[n]{x})^m$
	\end{itemize}

\textbf{exponential and logarithmic functions:}
	\begin{itemize}
		\item $\log_ax = y \leftrightarrow a^y = x$
		\item $\log_a(xy) = \log_ax + \log_ay$
		\item $\log_a(a^x) = x$
		\item $a^{\log_a x} = x$
		\item $\log_a(\frac{x}{y}) = \log_ax - \log_ay$
		\item $\log_a1 = 0$
		\item $\log_aa = 1$
		\item $\log_a(x^r) = r\log_ax$


			\textbf{insert change of base formulae here}
	\end{itemize}

\textbf{special factorizations:}
	\begin{itemize}
		\item $x^2 - y^2 = (x + y)(x - y)$
		\item $x^3 + y^3 = (x + y)(x^2 - xy + y^2)$
		\item $x^3 - y^3 = (x - y)(x^2 + xy + y^2)$
	\end{itemize}

\textbf{binomial, ect?}
	\begin{itemize}
		\item $(x + y)^2 = x^2 + 2xy + y^2$
		\item $(x - y)^2 = x^2 - 2xy + y^2$
		\item $(x + y)^3 = x^3 + 3x^2y + 3xy^2 + y^3$
		\item $(x - y)^3 = x^3 - 3x^2y + 3xy^2 - y^3$
		\item $(x + y)^n = x^n + nx^{n-1}y + \frac{n(n-1)}{2}x^{n-2}y^2 + \ldots + \binom{n}{k}x^{n-k}y^k + \dots + nxy^{n-1} + y^n$
			where $\binom{n}{k} = \frac{n(n-1) \dots (n-k+1)}{1 \cdot 2 \cdot 3 \cdot \ldots \cdot k}$
	\end{itemize}

\textbf{polynomials}
$a_nx^n + a_{n-1}x^{n-1} + \cdots + a_1x + a_0$\\
where:
	\begin{itemize}
		\item $a_n, a_{n-1}, \ldots, a_0$ are coefficients (real or complex)
		\item $x$ is the variable
		\item $n$ is a non-negative integer
		\item $a_n \neq 0$ (so the polynomial has degree $n$)
	\end{itemize}

so $\frac{1}{x} + 2$ and $\sqrt{x} + 1$ are not polynomials\\

domain: set of all input values for which the function is defined\\

range: set of all possible output values of the function\\

continuity: most algebaic functions are continuous (no breaks of jumps), but for example rational functions have discontinuities at points\\

behavior: the functions behavior is influenced by the degree of the polynomial and the nature of the function (consider even functions like $x^2$ or $x^4$ and their symmetry about the y-axis also consider odd functions like $x^3$ or $x^5$ that have point symmetry)\\

odd function: $f(-x) = -f(x)$\\
even function: $f(-x) = f(-x)$\\

vertical scaling (scaling along the y-axis): multiplying the output (y-values) of a function by a constant
	\begin{itemize}
		\item if $a > 1$: stretches the graph vertically (makes it taller)
		\item if $0 < a < 1$: compresses the graph vertical (makes it shorter)
		\item if $a < 0$ reflects the graph over the x-axis and scales it
	\end{itemize}

horizontal scaling (scaling along the x-axis): multiplying the input (x-values) by a constand inside the function
	\begin{itemize}
		\item if $b > 1$: compresses the graph horizontally (makes it narrower)
		\item if $0 < b < 1$: stretches the graph horizontally (makes it wider)
	\end{itemize}

consider $f(x) = x^2$:\\
$g(x) = 3x^2$ (vertical scaling)\\
$h(x) = (3x)^2$ (horizontal scaling) notice how here you will be squaring the scaling factor\\

BONUS $f(x) = x$ (consider why vertical and horizontal scaling looks the same for linear)\\

$\lvert a\rvert = \lvert -a\rvert$, $\lvert ab\rvert = \lvert a\rvert\lvert b\rvert$\\
The \textbf{distance} between two real numbers $a$ and $b$ is $\lvert b - a \rvert$, which is the length of the line segment joining $a$ and $b$.\\
Two real numbers $a$ and $b$ are close to each other if $\lvert b - a\rvert$ is small, and this is the case if their decimal expansions agree to many places. More precisely, \textit{if the decimal expansions of a and b agree to k places (to the right of the decimal point), then the distance $\lvert b - a\rvert$ is at most $10^-k$. Thus, the distance between a = 3.1415 and b = 3.1478 is at most $10^{-2}$ because a and b agree to two places. In fact, the distance is exactly $\lvert3.1415 - 3.1478\rvert = 0.0063$.}\\
Beware that $\lvert a + b\rvert$ is not equal to $\lvert a\rvert + \lvert b\rvert$ unless $a$ and $b$ have the same sign or at least one of $a$ and $b$ is zero. If they have opposite sins, cancellation occurs in the sum $a + b$ and $\lvert a+b\rvert < \lvert a\rvert + \lvert b\rvert$. For example, $\lvert 2 + 5\rvert = \lvert2\rvert + \lvert5\rvert$ but $\lvert-2 + 5\rvert = 3$, which is less than $\lvert-2\rvert + \lvert5\rvert = 7$. In any case, $\lvert a + b\rvert$ is never larger than $\lvert a\rvert + \lvert b\rvert$ and this gives us the simple but important \textbf{triangle inequality}: $\lvert a + b\rvert \leq \lvert a\rvert + \lvert b\rvert$\\

$[a, b] = {x \in \mathbb{R} : a \leq x \leq b}$\\
$(a, b) = {x \in \mathbb{R} : a < x < b}$\\
$[a, b) = {x \in \mathbb{R} : a \leq x < b}$\\
$(a, b] = {x \in \mathbb{R} : a < x \leq b}$\\
$(-r, r) = {x : \lvert x\rvert < r}$\\
$(c - a, c + a) = {\lvert x - c\rvert < a} = c - a < x < c + a$\\

\textbf{composing new functions}\\
	If $f$ and $g$ are functions, we may construct new functions by forming the sum, difference, product, and quotient functions:\\
	\begin{itemize}
		\item $(f + g)(x) = f(x) + g(x)$
		\item $(f - g)(x) = f(x) - g(x)$
		\item $(fg)(x) = f(x)g(x)$
		\item $(\frac{f}{g}(x) = \frac{f(x)}{g(x)})$
	\end{itemize}
	We can also multiply functions by constants. A function of the form: $c_1f(x) + c_2g(x)$ is called a \textbf{linear combination}.\\
	\textbf{Composition} is another important way of constructing new functions. The composition of $f$ and $g$ is the function $f \circ g$ defined by $(f \circ g)(x) = f(g(x))$, defined for values of $x$ in the domain of $g$ such that $g(x)$ lies in the domain of $f$.\\
	ex. Compute the composite functions $f \circ g$ and $g \circ f$ and discuss their domains where $f(x) = \sqrt{x}$ and $g(x) = 1 - x$\\
	solution: $(f \circ g)(x) = f(g(x)) = f(1 - x) = \sqrt{1 - x}$ The square root $\sqrt{1 - x}$ is defined if $1 - x \geq 0$ or $x \leq 1$, so the domain of $f \circ g$ is ${x : x \leq 1}$.\\
	On the other hand, $(g \circ f)(x) = g(f(x)) = g(\sqrt{x}) = 1 - \sqrt{x}$ The domain of $g \circ f$ is ${x : x \geq 0}$.\\

\textbf{invertable functions}\\
	"is this function invertible?" $\Leftrightarrow$ "does an inverse function exist for this function" $\Leftrightarrow$ "is the function one-to-one?" (horizontal line test)
	\begin{itemize}
		\item if it is, then the inverse function exists
		\item if it is not, then the inverse function does not exist, and the function is not invertible (as a function)
	\end{itemize}
	consider $f(x) = x^2$ this function is not one-to-one (horizontal line test) this it is not invertable unless you restrict the domain to be $x \geq 0$.\\
	to find the inverse algeraically you can swap the x's and y's and then solve for y\\

\textbf{rational functionas}\\
$f(x) = \frac{P(x)}{Q(x)}$ where $P(x)$ and $Q(x)$ are polynomials\\
$Q(x) \ne 0$\\
the domian is all real numbers except for where the denominator is 0\\
a vertial asymptote occurs where the denominator is zero (and not canceled by a common factor)\\
holes (removable discontinuities) occur if a factor cancels from both the numerator and denominator\\
horizontal asmptotes take $n$ to be the degree of the numerator and $m$ to be the degree if the denominator
	\begin{itemize}
		\item $n < m$: horizontal asymptote at $y = 0$
		\item $n = m$: horitzontal asymptote at $\frac{\text{leading coeff. of}P(x)}{\text{leading coeff. of}Q(x)}$
		\item $n > m$: no horizontal asymptote (however there may be an oblique/slant asymptote instead)
	\end{itemize}
slant(oblique) asymptotes occur when $n = m+1$ ... use polynomial division to find slant asymptotes
the $x$ intercepts occur where the numerator is zero (where does the function = 0)\\
the $y$ intercept: plug in $x = 0$\\

\textbf{conic sections}
	\begin{itemize}
		\item ellipses
			\begin{itemize}
				\item $(h, k)$ center of the ellipse
				\item $a$ semi-major axis (long radius)
				\item $b$ semi-minor axis (short radius)
				\item $c$ distance from center to each focus $c = \sqrt{a^2 - b^2}$
			\end{itemize}
			\begin{itemize}
				\item horizontal major axis $\frac{(x - h)^2}{a^2} + \frac{(y - k)^2}{b^2} = 1$
				\item vertical major axis $\frac{(x - h)^2}{b^2} + \frac{(y - k)}{a^2} = 1$
			\end{itemize}

		\item hyperbolas
			\begin{itemize}
				\item $(h, k)$ center
				\item $a$ distance from center to each vertex (on transverse axis) 
				\item $b$ related to the asymptotes
				\item $c = \sqrt{a^2 + b^2}$ distance from center to each focus (note here add not subtract like ellipse)
				\item asymptotes
					\begin{itemize}
						\item for horizontal hyperbola $y - k = \pm\frac{b}{a}(x - h)$
						\item for vertical hyperbola $y - k = \pm\frac{a}{b}(x - h)$
					\end{itemize}
				\item transverse axis: line through both vertices and foci
				\item conjugate axis: perpendicular to the transverse axis
			\end{itemize}
			\begin{itemize}
				\item opens horizontally $\frac{(x - h)^2}{a^2} - \frac{(y - k)^2}{b^2} = 1$  
				\item opens vertically $\frac{(x - h)^2}{b^2} - \frac{(y - k)}{a^2} = 1$ 
			\end{itemize}
	\end{itemize}

\textbf{complex numbers}
	$i = \sqrt{-1}, i^2 = -1$\\
	$z = a + bi$ rectangular form\\
	$z = r(\cos(\theta) + i\sin(\theta))$ where $r = \lvert z\rvert$ and $\theta = \tan^{-1}(\frac{b}{a})$\\

	$(2 + 3i)(1 - 4i) = 2 - 8i + 3i - 12i^2 = (14 - 5i)$\\
	$\frac{1 + i}{2 - 3i} = \frac{(1 + i)(2 + 3i)}{(2 - 3i)(2 + 3i)} = \frac{2 + 5i - 3}{4 + 9} = -\frac{1}{13} + \frac{5}{13}i$\\	
	$a^2 + b^2 = (a + bi)(a - bi)$\\

	multiplying by i has a very cool and geometric meaning in the complex plane: it corresponds to a 90-degree counterclockwise rotation about the origin.\\

	$i(a + bi) = ai + bi^2 = ai - b = -b + ai$ so $i(a + bi) = -b + ai$ that is a new complex number whose real part is -b, and imaginary part is a\\

\textbf{scalars and vectors}
	\begin{itemize}
		\item a scalar is a quantity that has only magnitude (size or amount) think temperature, mass, time, speed
		\item a vector has both a magnitude and a direction... think displacement, velocity, force
	\end{itemize}

	find the component form of $\vec{a}$ given:\\
	$\lvert\vec{a}\rvert = 3$ and $\theta = 30^{\circ}$\\
	$\vec{a} = (3\cos(30^{\circ}), 3\sin(30^{\circ}))$\\
	$\vec{a} = (\frac{3\sqrt{3}}{2}, \frac{3}{2})$\\

	now find the magnitude and direction of $\vec{b} = (\sqrt{2}, \sqrt{2})$\\
	$\lvert\vec{b}\rvert = \sqrt{(\sqrt{2})^2 + (\sqrt{2})^2} = 2$\\
	$\theta = \tan^{-1}(\frac{\sqrt{2}}{\sqrt{2}}) = 45^{\circ}$ (be aware of the quadrants)\\

	$\vec{w} = (1, 2)$ so $3\vec{w} = (3, 6)$\\

	$\vec{a} = (3, -1)$ and $\vec{b} = (2, 3)$\\
	$\vec{a} + \vec{b} = (3+2, -1+3) = (5, 2)$
	$\vec{a} - \vec{b} = (3-2, -1-3) = (1, -4)$ (you could also just add the negative of $\vec{b}$)\\

\textbf{matrices}\\
$
\begin{bmatrix}
a & b\\
c & d
\end{bmatrix}
$\\

\textbf{probability and combinatorics}
consider a standard deck of cards (no jokers)\\
$P(jack) = \frac{4}{52} = \frac{1}{13}$\\
$P(hearts) = \frac{13}{52}$\\

now consider a venn diagram for the following:
$P(J \cup H) = \frac{4 + 13 - 1}{52}$\\
since the probability of hearts overlaps with the probability of jacks (jack of hearts) youll have to subtract out the double counting...we say that that are not mutually exclusive otherwise you could just add $P(J) + P(H)$ if they were mutually exclusive\\

addition rule:\\
$P(A \cup B) = P(A) + P(B) - P(A \cap B)$\\

now consider mutually exclusive events:\\
$P(A \cap B) = 0$\\
$P(A \cup B) = P(A) + P(B)$\\

multiplication rule for independent events:\\
if two events, A and B, are independent (meaning the occurrence of one does not affect the other), then: $P(A \cap B) = P(A) \cdot P(B)$\\

consider a fair coin:\\
$P(HH) = \frac{1}{2} \cdot \frac{1}{2}$

now consider dependent events say you have a bag with 3 blue and 2 red balls...what is the probability that the first pull is blue and the second is blue\\
$P(1st blue) \cdot P(2nd blue \mid 1st blue)$\\

$P(A \cap B) = P(B) \cdot P(A \mid B) = P(A) \cdot P(B \mid A)$\\

permutations (order matters)\\
$P(n, k) = \frac{n!}{(n - k)!}$\\

combinations (order does not matter)\\
$\binom{n}{k} = \frac{n!}{k!(n - k)!}$\\

a probability distribution describes how likely different outcomes are for a random variable
	\begin{itemize}
		\item a random variable is something that can take on different values due to chance (like rolling a die or counting the number of heads in coin flips)
		\item a probability distribution assigns a probability to each possible value that the variable can take
	\end{itemize}

the expected value of a random variable is the long-run average outcome you would expect if you repeated an experiment many times. think of it as the center or balance point of a probability distribution.\\

$E[X] = \sum_{i=1}^{n} x^i \cdot P(x_i)$\\

\textbf{sequences, series}\\
an arithmetic series is the sum of the terms in an arithmetic sequence, where each term increases or decreases by a constant value (common difference)\\
nth term:\\
$a_n = a_1 + (n-1)d$
sum of the first n terms:\\
$S_n = \frac{n}{2}(a_1 + a_n)$\\
$S_n = \frac{n}{2}(2a_1 + (n-1)d)$\\
$a, a + d, a + 2d, \dots, a + (n - 1)d$\\
the sum of the first $n$ terms: $\sum_{k=0}^{n-1} (a + kd)$\\

a geometric series is the sum of the terms in a geometric sequence, where each term is multiplied by a constant (common ration)\\
nth term:\\
$a_n = a_1 \dot r^{n-1}$\\
sum of the first n terms (when $r \neq 1$):\\
$S_n = a_1 \cdot \frac{1 - r^n}{1 - r}$\\
if $\lvert r\rvert < 1$ and $n \to \infty$, then the infinite sum:\\
$S = \frac{a_1}{1 - r}$
$a, ar, ar^2, \dots, ar^{n-1}$\\
the sum of the first n elements: $\sum_{k=0}^{n-1} (ar^k)$\\

the binomial theorem is a powerful tool to expand expressions of the form $(a + b)^n$ where $a$ and $b$ are any numbers or variables and $n$ is non-negative\\
$(a + b)^n = \sum_{k=0}^{n} \binom{n}{k}a^{n-k}b{k}$\\

pascal's triangle is a triangular array of numbers that shows the binomial coeffiecients - which appear in the binomial therem expansions\\

each number is the sum of the two numbers directly above it $\binom{n}{k} = \binom{n-1}{k-1} + \binom{n-1}{k}$ where for $\binom{n}{k}$ the value at row $n$, position $k$, the rows start at $n=0$, each row gives the coefficients for $(a+b)^n$

\newpage
\section*{trigonometry}

$
\begin{array}{|c|c|c|c|}
\hline
\text{Angle (Degrees)} & \text{Angle (Radians)} & \cos(\theta) & \sin(\theta) \\
\hline
0^\circ & 0 & 1 & 0 \\
30^\circ & \frac{\pi}{6} & \frac{\sqrt{3}}{2} & \frac{1}{2} \\
45^\circ & \frac{\pi}{4} & \frac{\sqrt{2}}{2} & \frac{\sqrt{2}}{2} \\
60^\circ & \frac{\pi}{3} & \frac{1}{2} & \frac{\sqrt{3}}{2} \\
90^\circ & \frac{\pi}{2} & 0 & 1 \\
\hline
120^\circ & \frac{2\pi}{3} & -\frac{1}{2} & \frac{\sqrt{3}}{2} \\
135^\circ & \frac{3\pi}{4} & -\frac{\sqrt{2}}{2} & \frac{\sqrt{2}}{2} \\
150^\circ & \frac{5\pi}{6} & -\frac{\sqrt{3}}{2} & \frac{1}{2} \\
180^\circ & \pi & -1 & 0 \\
\hline
210^\circ & \frac{7\pi}{6} & -\frac{\sqrt{3}}{2} & -\frac{1}{2} \\
225^\circ & \frac{5\pi}{4} & -\frac{\sqrt{2}}{2} & -\frac{\sqrt{2}}{2} \\
240^\circ & \frac{4\pi}{3} & -\frac{1}{2} & -\frac{\sqrt{3}}{2} \\
270^\circ & \frac{3\pi}{2} & 0 & -1 \\
\hline
300^\circ & \frac{5\pi}{3} & \frac{1}{2} & -\frac{\sqrt{3}}{2} \\
315^\circ & \frac{7\pi}{4} & \frac{\sqrt{2}}{2} & -\frac{\sqrt{2}}{2} \\
330^\circ & \frac{11\pi}{6} & \frac{\sqrt{3}}{2} & -\frac{1}{2} \\
360^\circ & 2\pi & 1 & 0 \\
\hline
\end{array}
$\\

$\pi$ radians = $180^{\circ}$\\
$1^\circ = \frac{\pi}{180}$ radians\\
$1$ radian = $\frac{180^\circ}{\pi}$\\

Trigonometric functions are special mathematical functions that originally come from studying right triangles. They relate the size of an angle in a triangle to the ratios of the lengths of the triangle's sides.\\

\textbf{SOH-CAH-TOA} is a mnemonic device that expresses the relationship between the basic trigonometric functions and the ratios of the sides in a right triangle.\\

The triangle definition only works for angles between 0 and $\frac{\pi}{2}$. To extend trig functions to all angles (including negative angles and angles larger then ($2\pi$ or $360^{\circ}$), mathematicians use the unit circle. Thus allowing use to use trig functions on the coordinate plane, enabling graphing and calculus.\\

To derive the rest of the fundamental trigonometric identities, you need a combination of a few key identities and principles. The most important starting point is the Pythagorean identity, but you’ll also need the basic relationships between the trigonometric functions, such as the definitions of sine, cosine, tangent, secant, cosecant, and cotangent in terms of a right triangle or the unit circle.
	\begin{itemize}
		\item $\frac{1}{\cos(\theta)} = \sec(\theta)$	
		\item $\frac{1}{\sin(\theta)} = \csc(\theta) $
		\item $\frac{1}{\tan(\theta)} = \cot(\theta)$
		\item $\sin(\theta + 2\pi) = \sin(\theta)$
		\item $\cos(\theta + 2\pi) = \cos(\theta)$
		\item $\tan(\theta + \pi) = \tan(\theta)$
		\item $\cos(\frac{\pi}{2} - \theta) = \sin(\theta)$
		\item $\sin(\frac{\pi}{2} - \theta) = \cos(\theta)$
		\item $\tan(\frac{\pi}{2} - \theta) = \cot(\theta)$
		\item $\sin(-\theta) = -\sin(\theta)$
		\item $\cos(-\theta) = \cos(\theta)$
		\item $\tan(\theta) = -\tan$
		\item $\sin^2(\theta) + \cos^2(\theta) = 1$
		\item $1 + \tan^2(\theta) = \sec^2(\theta)$
		\item $1 + \cot^2(\theta) = \csc^2(\theta)$
		\item $\cos(A \pm B) = \cos(A)\cos(B) \mp \sin(A)\sin(B)$		
		\item $\sin(A \pm B) = \sin(A)\cos(B) \pm \cos(A)\sin(B)$
		\item $\tan(A \pm B) = \frac{\tan(A) \pm \tan(B)}{1 \mp \tan(A)\tan(B)}$
		\item $\sin(2\theta) = 2\sin(\theta)\cos(\theta)$
		\item $\cos(2\theta) = \cos^2(\theta) - \sin^2(\theta)$		
		\item $\cos(2\theta) = 2\cos^2(\theta) - 1$
		\item $\cos(2\theta) = 1 - 2\sin^2(\theta)$
	\end{itemize}



\textbf{law of sines and cosines}\\

	$\frac{\sin(A)}{a} = \frac{\sin(B)}{b} \frac{\sin(C)}{c}$\\
	$a^2 = b^2 + c^2 - 2bc\cos(A)$\\

\newpage
\section*{limits}
	so let us consider $f(x) = \frac{\sin(x)}{x}$ ($x$ in radians)\\
	the value of $f(0)$ is not defined because\\
	$f(0) = \frac{\sin(0)}{0} = \frac{0}{0}$\\
	however if we take $x \to 0^{+}$ and $x \to 0^{-}$ we get the impression that $f(x)$ gets closer and closer to 1 as $x \to 0$ from the left and right.
	\begin{center}
		$\lim_{x \to 0}f(x) = 1$
	\end{center}
	recall that the distance between two numbers $\lvert a - b\rvert$...thus we can express the idea that $f(x)$ is close to $L$ by saying that $\lvert f(x) - L\rvert$ is small.\\

	\textbf{informal limit definition} Assume that $f(x)$ is defined for all $x$ near $c$ (i.e., in some open interval containing c), but not necessarily at $c$ itself. We say that:\\
	\textit{the limit of $f(x)$ as $x$ approaches $c$ is equal to $L$}\\
	if $\lvert f(x) - L\rvert$ becomes arbitarily small when $x$ is any number sufficiently close (but not equal) to $c$. In this case, we write:
	\begin{center}
		$\lim_{x \to c}f(x) = L$
	\end{center}
	we also say the $f(x)$ approaches or converges to $L$ as $x \to c$ (and we write $f(x) \to L$)\\

	If the values of $f(x)$ do not converge to any limit as $x \to c$, we say that $\lim_{x \to c}f(x)$ DNE. It is important to note that the value $f(c)$ itself, which may or may not be defined, plays no role in the limit. All that matters are the values $f(x)$ for $x$ close to c. Furthermore, if $f(x)$ approaches a limit as $x \to c$, then the limiting value $L$ is unique.\\

	We say the $\lim_{x \to c}f(x) = \infty$ if $f(x)$ increases beyond bound as $x$ approaches $c$, and $\lim_{x \to c}f(x) = -\infty$ if $f(x)$ becomes arbitrarily large (in absolute value) but negative as $x$ approaches $c$\\

	The Limit Laws state that if $lim_{x \to c}f(x)$ and $lim_{x \to c}g(x)$ both exist, then
		\begin{itemize}
			\item $lim_{x \to c}(f(x) + g(x)) = lim_{x \to c}f(x) + lim_{x \to c}g(x)$
			\item $lim_{x \to c}kf(x) = klim_{x \to c}f(x)$
			\item $lim_{x \to c}f(x)g(x) = (lim_{x \to c}f(x))(lim_{x \to c}g(x))$
			\item if $lim_{x \to c}g(x) \neq 0$, then $lim_{x \to c}\frac{f(x)}{g(x)} = \frac{lim_{x \to c}f(x)}{lim_{x \to c}g(x)}$
			\item if $lim_{x \to c}f(x)$ or $lim_{x \to c}g(x) DNE$, then the Limit Laws cannot be applied
		\end{itemize}

	Assume that $f(x)$ is defined on an open interval containing $x = c$. Then $f$ is continuous at $x = c$ if $lim_{x \to c}f(x) = f(c)$ If the limit does not exist, or if it exists but is not equal to $f(c)$, we say that $f$ has a discontinuity (or is discontinuous) at $x = c$.\\

	A function $f(x)$ may be continuous at some points and discontinuous at others. if $f(x)$ is continuous at all points in an interval $I$, then $f(x)$ is said to be continuous on $I$. Here, if I is an interval $[a, b]$ or $[a, b)$ that includes $a$ as a left-endpoint, we require that $lim_{x \to a+}f(x) = f(a)$. Similarly, we require that $lim_{x \to b-}f(x) = f(b)$ if $I$ includes $b$ as a right-endpoint $b$. If $f(x)$ is continuous at all points in its domain, then $f(x)$ is simply called continuous.\\

	Lets look at how a function can fail to be continuous...remeber that point discontinuity requires that the limit exist at that point, the value of the function exists at that point, and those two equal. If the first two conditions hold but that last one fails then we say that the function has a \textbf{removable discontinuity} at that point.\\

	Removable discontinuities are "mild" in the following sense: We can make $f$ continuous at $x = c$ by redefining $f(c)$.\\

	A "worse" type of discontinuity is a \textbf{jump discontinuity}, which occurs if the one-sided limits $lim_{x \to c-}f(x)$ and $lim_{x \to c+}f(x)$ exist but are not equal.\\

	We say that $f(x)$ has an infinite discontinuity at $x = c$ if one or both of the one-sided limits is infinite (even if $f(x)$ itself is not defined at $x = c$).\\

	We should mention that some functions have more "severe" types of discontinuity than those discussed above. For example, $f(x) = \sin(\frac{1}{x})$ oscillates infinitely often between $+1$ and $-1$ as $x \to 0$. Nether the left-nor the right-hand limits exist at $x = 0$, so this discontinuity is no a jump discontinuity. Although of interest from a theoretical point of view, these discontinuities rarely arise in practice.\\

	\textbf{Laws of Continuity}\\
	Assume that $f(x)$ and $g(x)$ are continuous at a point $x = c$. Then the following functions are also continuous at $x = c$:\\
		\begin{itemize}
			\item $f(x) + g(x) and f(x) - g(x)$
			\item $kf(x)$ for any constant $k$
			\item $f(x)g(x)$
			\item $\frac{f(x)}{g(x)}$ if $g(c) \neq 0$
		\end{itemize}

	\textbf{Continuity of Polynomial and Rational Functions}\\
	Let $P(x)$ and $Q(x)$ be polynomials. Then:\\
		\begin{itemize}
			\item $P(x0 is continuous on the real line$
			\item $\frac{P(x)}{Q(x)}$ is continuous at all values $c$ such that $Q(x) \neq 0$
		\end{itemize}

	\textbf{Continuity of Some Basic Functions}\\
		\begin{itemize}
			\item $y = \sin(x)$ and $y = \cos(x)$ are continuous on the real line
			\item For $b > 0$, $y = b^x$ is continuous on the real line
			\item If $n$ is a natural number, then $y = x^{\frac{1}{n}}$ is continuous on its domain
		\end{itemize}

	\textbf{Continuity of Composite Functions}\\
		Let $F(x) = f(g(x))$ be a composite function. If $g$ is continuous at $x = c$ and $f$ is continuous at $x = g(c)$, then $F(x)$ is continuous at $x = c$\\	

		It is easy to evaluate a limit when the function in question is known to be continuous. in this case, by definition, the limit is equal to the function value.\\

		In general, we say that $f(x)$ has an \textbf{indeterminate form} at $x = c$ if, when $f(x)$ is evaluated at $x = c$, we obtain an indefined expression of the type $\frac{0}{0}$, $\frac{\infty}{\infty}$, $\infty \cdot 0$, $\infty - \infty$	\\

Later when we study derivatives, we will be faced with limits $lim_{x \to c}f(x)$, where $f(c)$ is not defined. In such cases, substitution cannot be used directly. However, some of these limits can be evaluated using substitution, provided that we first use algebra to rewrite the formula for $f(x)$\\

	\textbf{Evaluating Limits Algebraically}
		\begin{itemize}
			\item $lim_{x \to 4}\frac{x^2 - 16}{x - 4}$\\
				The function is not defined at $x = 4$ because $f(4) = \frac{4^2 - 16}{4 - 4}  = \frac{0}{0}$ (arithmetically undefined)\\

				However, the numerator of $f(x)$ factors and\\
				$\frac{x^2 - 16}{x - 4} = \frac{(x+4)(x-4)}{x - 4} = x + 4$ (valid for $x \neq 4$)\\
				In other words, $f(x)$ coincides with the \textit{continuous} function $x + 4$ for all $x \neq 4$. Sinve the limit depends only on the values of $f(x)$ for $x \neq 4$, we have\\
				$lim_{x \to 4}\frac{x^2 - 16}{x - 4} = lim_{x \to 4}(x + 4) = 8$\\

				In general, we say that $f(x)$ has an \textbf{indeterminate form)} at $x = c$, we obtain an undefined expression of the type\\
				
				$\frac{0}{0}, \frac{\infty}{\infty}, \infty \cdot 0, \infty - \infty$\\

				We also say that $f$ is \textbf{indeterminate} at $x = c$. Our strategy is to \textit{transform $f(x)$ algebraically if possible into a new expression that is defined and continuous at $x = c$, and then evaluate by substitution ("plugging in")}. As you study the following examples, notice that the critical step in each case is to cancel a common factor from the numerator and denominator at the appropriate moment, thereby removing the indeterminacy.
			\item $lim_{x \to 4}\frac{\sqrt{x} - 2}{x - 4}$\\
				The function $f(x) = \frac{\sqrt{x} - 2}{x - 4}$ is indeterminate at $x = 4$ since\\

				$f(4) = \frac{\sqrt{4} - 2}{4 - 4} = \frac{0}{0}$ (indeterminate)\\

				$(\frac{\sqrt{x} - 2}{x - 4})(\frac{\sqrt{x} + 2}{\sqrt{x} + 2}) = \frac{x - 4}{(x - 4)(\sqrt{x} + 2)} = \frac{1}{\sqrt{x} + 2}$ (if $x \neq 4$)\\

				Since $\frac{1}{\sqrt{x} + 2}$ is continuous at $x = 4$,\\

				$lim_{x \to 4}\frac{\sqrt{x} - 2}{x - 4} = lim_{x \to 4}\frac{1}{\sqrt{x} + 2} = \frac{1}{4}$

			\item $lim_{x \to 2}\frac{x^2 - x + 5}{x - 2}$\\
				At $x = 2$ we have\\
				
$f(2) = \frac{2^2 - 2 + 5}{2 - 2} = \frac{7}{0}$ (undefined, but not an indeterminate form)\\
				This is \textit{not} an indeterminate form. In fact, shows that the one-sided limits are infinte:\\

				$lim_{x \to 2-}\frac{x^2 - x + 5}{x - 2} = -\infty$, $lim_{x \to 2+}\frac{x^2 - x + 5}{x - 2} = \infty$\\

				The limit itself does not exist.
		\end{itemize}

	\textbf{The Squeeze Theorem}\\
	In our study of the derivative, we will need to evaluate certain limits involving transcendental functions such as sine and cosine. The algebraic techniques of the previous section are often ineffective for such functions and other tools are required. One such tool is the Squeeze Theorem, which we discuss in this section and use to evaluate the trigonometric limits later.\\

	Consider a function $f(x)$ that is trapped between two functions $l(x)$ and $u(x)$ on an interval $I$. In other words,\\

	$l(x) \leq f(x) \leq u(x)$ for all $x \in I$\\

	In this case, the graph of $f(x)$ lies between the graphs of $l(x)$ and $u(x)$, with $l(x)$ as the lower and $u(x)$ as the upper function.\\

	The Squeeze Theorem applies when $f(x)$ is not just trapped, but actually \textbf{squeezed} at a point $x = c$ by $l(x)$ and $u(x)$. By this we mean that for all $x \neq c$ in some open interval containing $c$,\\

	$l(x) \leq f(x) \leq u(x)$ and $lim+{x \to c}l(x) = lim_{x \to c}u(x) = L$\\

	We do not require that $f(x)$ be defined at $x = c$, but it is clear graphically that $f(x)$ must approach the limit $L$. We state this formally:\\

	Assume that for $x \neq c$ (in some open interval containing $c$),\\

	$l(x) \leq f(x) \leq u(x)$ and $lim_{x \to c}l(x) = lim{x \to c}u(x) = L$\\

	Then $lim_{x \to c}f(x)$ exists and\\

	$lim_{x \to c}f(x) = L$\\

	\textbf{two important limits}\\
	$lim_{\theta \to 0}\frac{\sin(\theta)}{\theta} = 1$\\
	$lim_{\theta \to 0}\frac{1 - \cos{\theta}}{\theta} = 0$\\

	The \textbf{Intermediate Value Theorem} is a basic result which states that \textit{a continuous function on an interval cannot skip values.}\\
	
	If $f(x)$ is continuous on a closed interval $[a, b]$ and $f(a) \neq f(b)$, then for every value $M$ between $f(a)$ and $f(b)$, there exists at least one value $c \in (a, b)$ such that $f(c) = M$\\

	Prove that the equation $\sin(x) = 0.3$ has at least one solution.\\

	We may apply the IVT since $\sin(x)$ is continuous. We choose an interval where we suspect that a solution exists. The desired value 0.3 lies between the two function value $\sin(0) = 0$ and $\sin(\frac{\pi}{2} = 1$\\

	so the interval $[0 , \frac{\pi}{2}]$ will work. The IVT tells us that $\sin(x) = 0.3$ has at least one solution in $(0, \frac{\pi}{2})$. Since $\sin(x)$ is periodic, $\sin(x)$ actually has infinitely many solutions.\\

	The IVT can be used to show the existence of zeros of functions. If $f(x)$ is continuous and takes on both positive and negative values, say, $f(a) < 0$ and $f(b) > 0$, then the IVT guarantees that $f(c) = 0$ for some $c$ between $a$ and $b$.\\

	If $f(x)$ is continuous on $[a, b]$ and if $f(a)$ and $f(b)$ are nonzero and have opposite signs, then $f(x)$ has a zero in $(a, b)$.\\

	We can locate zeros of functions to arbitrary accuracy using the \textbf{Bisection Method}.\\

	Show that $f(x) = \cos^2(x) - 2\sin(\frac{x}{4})$ has a zero in $(0, 2)$. Then locate the zero more accurately using the Bisecition Method.\\

	Using a calculator, we find that $f(0)$ and $f(2)$ have opposite signs: $f(0) = 1 > 0$, $f(2) \approx -0.786 < 0$\\

	We can locate a zero more accurately by dividing $[0, 2]$ into two intervals $[0, 1]$ and $[1, 2]$. One of these must contain a zero of $f(x)$. To determine which, we evaluate $f(x)$ at the midpoint $m = 1$. A calculator gives $f(1) \approx -0.203$, and since $f(0) = 1$, we see that\\

	$f(x)$ takes on opposite signs at the endpoints of $[0, 1]$\\

	Therefore, $(0, 1)$ must contain a zero. We discard the $[1, 2]$ because both $f(1)$ and $f(2)$ are negative. The Bisection Method consists of continuing this process until we narrow down the location of the zero to the desired accuracy.\\

	The IVT seems to state the obvious, namely that a continuous function cannot skip values. Yet its proof is quite subtle because it depends on the \textit{completeness property} of the real numbers. To highlight the subtlety observe that IVT is false for functions defined only on the \textit{rational numbers}. For example, $f(x) = x^2$ does not have the intermediate value property if we restrict its domain to the rational numbers. Indeed, $f(0) = 0$ and $f(2) = 4$ but $f(c) = 2$ \textit{has no solution} for $c$ rational. The solution $c = \sqrt{2}$ is "missing" from the set of rational numbers because it is irrational. From the beginnings of calculus, the IVT was surely regarded as obvious. However, it was not possible to give a genuinely rigorous proof until the completeness property was clarified in the second half of the ninteenth century.\\

	\textbf{The Size of the Gap}\\
	Recall that the distance from $f(x)$ to $L$ is $\lvert f(x) - L\rvert$. It is convenient to refer to that quantity $\lvert f(x) - L\rvert$ as the \textit{gap} between the value $f(x)$ and the limit $L$.\\

	Let us reexamine the basic trigonometric limit $lim_{x \to 0}\frac{\sin(x)}{x} = 1$\\

	In this example, $f(x) = \frac{\sin(x)}{x}$ and $L = 1$, so (1) tells us that the gap $\lvert f(x) - 1\rvert$ gets arbitrarily small when $x$ is sufficiently close but not equal to 0.\\
	Suppose we want the gap $\lvert f(x) - 1\rvert$ to be less than 0.2. How close to 0 must $x$ be?\\
	The following statement is true: $\lvert \frac{\sin(x)}{x} - 1\rvert < 0.2$ if $0 < \lvert x\rvert < 1$\\

	If we insist instead that the gap be smaller than 0.004... $\lvert \frac{\sin(x)}{x} - 1 < 0.004$ if $0 < \lvert x\rvert$\\

	It would seem that this process can be continued: By zooming in on the graph, we can find a small interval around $c = 0$ where the gap $\lvert f(x) - 1\rvert$ is smaller than any prescribed positive number.\\
	To express this in a precise fashion, we follow time-honored tradition and use the Greek letter $\epsilon$ and $\delta$ to denote small numbers specifying the size of the gap and the quantity $\lvert x - c\rvert$, respectively. In our case, $c = 0$ and $\lvert x - c\rvert$ = $\lvert x - 0\rvert$ = $\lvert x\rvert$. The precise meaning is that for every choise of $\epsilon > 0$, there exists some $\delta$ (depending on $\epsilon$) such that $\lvert \frac{\sin(x)}{x} - 1\rvert < \epsilon$ if $0 < \lvert x\rvert < \delta$\\

	The number $\delta$ tells us how close is sufficiently close for a given $\epsilon$. With this motivation, we are ready to state the formal definition of the limit.\\

	\textbf{FORMAL DEFINITION OF A LIMIT} Suppose the $f(x)$ is defined for all $x$ in an open interval containing $c$ (but no necessarily at $x = c$). Then \begin{center}$lim_{x \to c}f(x) = L$\end{center} if for all $\epsilon > 0$, there exists $\delta > 0$ such that \begin{center}$\lvert f(x) - L\rvert<\epsilon$ if $0<\lvert x - c\rvert<\delta$\end{center} The condition $0 < \lvert x - c\rvert < \delta$ in this definition excludes $x = c$. As in our previous informal definition, we formulate it this way so that the limit depends only on values of $f(x)$ near $c$ but no on $f(c)$ itself. As we have seen, in many cases the limit exists even when $f(c)$ is not defined.

\newpage
\section*{differentiation}

\begin{center} $f'(a) = lim_{h \to 0}\frac{f(a + h) - f(a)}{h}$ \end{center}

$\frac{d}{dx}(c) = 0$\\
$\frac{d}{dx}x = 1$\\
$\frac{d}{dx}(x^n) = nx^{-1}$ (power rule)\\
$\frac{d}{dx}[cf(x)] = cf'(x)$\\
$\frac{d}{dx}[f(x)+g(x)] = f'(x) + g'(x)$\\
$\frac{d}{dx}[f(x)g(x)] = f(x)g'(x) + g(x)f'(x)$ (product rule)\\
$\frac{d}{dx}[\frac{f(x)}{g(x)}] = \frac{g(x)f'(x) - f(x)g'(x)}{[g(x)]^2}$ (quotient rule)\\
$\frac{d}{dx}f(g(x)) = f'(g(x))g'(x)$ (chain rule)\\
$\frac{d}{dx}f(x)^n = nf(x)^{n-1}f'(x)$ (general power rule)\\
$\frac{d}{dx}\sin(x) = \cos(x)$\\ 
$\frac{d}{dx}\cos(x) = -\sin(x)$\\
$\frac{d}{dx}\tan(x) = \sec^2(x)$
$\frac{d}{dx}\csc(x) = -\csc(x)\cot(x)$\\
$\frac{d}{dx}\sec(x) = \sec(x)\tan(x)$\\
$\frac{d}{dx}\cot(x) = -\csc^2(x)$\\
$\frac{d}{dx}\sin^{-1}(x) = \frac{1}{\sqrt(1 - x^2)}$\\
$\frac{d}{dx}\cos^{-1}(x) = -\frac{1}{\sqrt(1 - x^2)}$\\
$\frac{d}{dx}\tan^{-1}(x) = \frac{1}{1 + x^2}$\\
$\frac{d}{dx}(e^x) = e^x$\\
$\frac{d}{dx}(a^x) = (\ln a)a^x$\\
$\frac{d}{dx}\ln\mid x\mid = \frac{1}{x}$\\
$\frac{d}{dx}\log_ax = \frac{1}{(\ln a)x}$\\

\section*{integration}



\section*{review plan}
factor some quadratics\\
practice some trig identity manipulations\\
solving basic equations and sinusoidal equations\\
review (some) linear algebra\\
solve limits (basic algebraic manipulation), trig, limits at infty\\
some $\delta \epsilon$ proofs\\

review your sequences and series section above\^\\

geometrically understand sectors $S = r\theta$, $A = \frac{1}{2}r^2\theta$, and area of triangle in this form $\frac{1}{2}ab\sin(\theta)$\\ 

add $f'(a) = lim_[x \to a]\frac{f(x) - f(a)}{x - a}$ to differentiation section...\\
understand formal tangent line definition $y - f(a) = f'(a)(x - a)$\\
review power, product, quotient, chain rule\\
implicit differentiation\\
EVT\\
rolle's theorem\\
MVT\\
newtons methods\\

antiderivatives and integration\\

\end{document}
