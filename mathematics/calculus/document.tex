\documentclass{article}
\usepackage{amsmath}
\usepackage{amssymb}
\setlength{\parindent}{0pt}

\usepackage{graphicx}
\usepackage[a4paper, left=1.5in, right=1.5in, top=1.5in, bottom=1.5in]{geometry}
\usepackage{xcolor}

\usepackage{pgfplots}
\usetikzlibrary{intersections,angles,quotes}
\usepgfplotslibrary{fillbetween}

\title{calculus review document}

\date{\today}

\begin{document}
\maketitle

\section*{introduction}

	midpoint: $(\frac{x_1 + x_2}{2}, \frac{y_1 + y_2}{2})$\\
	distance: $\sqrt{(x_2 - x_1)^2 + (y_2 - y_1)^2}$\\
	circle: $(x - h)^2 + (y - k)^2 = r^2$\\
	pythagorean: $c^2 = a^2 + b^2$\\

	triangle\\
	circle\\
	sector of circle\\
	sphere\\
	cylinder\\
	cone\\
	cone with arbitrary base\\

\textbf{laws of exponents:} 
	\begin{itemize}
		\item $x^mx^n = x^{m+n}$
		\item $\frac{x^m}{x^n} = x^{m-n}$
		\item $(x^m)^n = x^{mn}$
		\item $x^{-n} = \frac{1}{x^n}$
		\item $(xy)^n = x^ny^n$
		\item $(\frac{x}{y})^n = \frac{x^n}{y^n}$ \item $x^{1/n} = \sqrt[n]{x}$
		\item $x^{\frac{m}{n}} = \sqrt[n]{x^m} = (\sqrt[n]{x})^m$
	\end{itemize}

\textbf{exponential and logarithmic functions:}
	\begin{itemize}
		\item $\log_ax = y \leftrightarrow a^y = x$
		\item $\ln x = y \leftrightarrow e^y = x$
		\item $\log_a(xy) = \log_ax + \log_ay$
		\item $\log_a(a^x) = x$
		\item $a^{\log_a x} = x$
		\item $\ln(e^x) = x e^{\ln x} = x$
		\item $\log_a(\frac{x}{y}) = \log_ax - \log_ay$
		\item $\log_a1 = 0$
		\item $\log_aa = 1$
		\item $\ln1 = 0$
		\item $\ln e = 1$
		\item $\log_a(\frac{x}{y}) = \log_ax - \log_ay$
		\item $\log_a1 = 0$
		\item $\log_aa = 1$
		\item $\ln1 = 0$
		\item $\ln e = 1$
		\item $\log_a(x^r) = r\log_ax$
	\end{itemize}

\textbf{special factorizations:}
	\begin{itemize}
		\item $x^2 - y^2 = (x + y)(x - y)$
		\item $x^3 + y^3 = (x + y)(x^2 - xy + y^2)$
		\item $x^3 - y^3 = (x - y)(x^2 + xy + y^2)$
	\end{itemize}

\textbf{binomial}
	\begin{itemize}
		\item $(x + y)^2 = x^2 + 2xy + y^2$
		\item $(x - y)^2 = x^2 - 2xy + y^2$
		\item $(x + y)^3 = x^3 + 3x^2y + 3xy^2 + y^3$
		\item $(x - y)^3 = x^3 - 3x^2y + 3xy^2 - y^3$
		\item $(x + y)^n = x^n + nx^{n-1}y + \frac{n(n-1)}{2}x^{n-2}y^2 + \ldots + \binom{n}{k}x^{n-k}y^k + \dots + nxy^{n-1} + y^n$
			where $\binom{n}{k} = \frac{n(n-1) \dots (n-k+1)}{1 \cdot 2 \cdot 3 \cdot \ldots \cdot k}$
	\end{itemize}

\textbf{polynomials}
$a_nx^n + a_{n-1}x^{n-1} + \cdots + a_1x + a_0$\\
where:
	\begin{itemize}
		\item $a_n, a_{n-1}, \ldots, a_0$ are coefficients (real or complex)
		\item $x$ is the variable
		\item $n$ is a non-negative integer
		\item $a_n \neq 0$ (so the polynomial has degree $n$)
	\end{itemize}

so $\frac{1}{x} + 2$ and $\sqrt{x} + 1$ are not polynomials\\

domain: set of all input values for which the function is defined\\

range: set of all possible output values of the function\\

continuity: most algebaic functions are continuous (no breaks of jumps), but for example rational functions have discontinuities at points\\

behavior: the functions behavior is influenced by the degree of the polynomial and the nature of the function (consider even functions like $x^2$ or $x^4$ and their symmetry about the y-axis also consider odd functions like $x^3$ or $x^5$ that have point symmetry)\\

odd function: $f(-x) = -f(x)$\\
even function: $f(-x) = f(-x)$\\

vertical scaling (scaling along the y-axis): multiplying the output (y-values) of a function by a constant
	\begin{itemize}
		\item if $a > 1$: stretches the graph vertically (makes it taller)
		\item if $0 < a < 1$: compresses the graph vertical (makes it shorter)
		\item if $a < 0$ reflects the graph over the x-axis and scales it
	\end{itemize}

horizontal scaling (scaling along the x-axis): multiplying the input (x-values) by a constand inside the function
	\begin{itemize}
		\item if $b > 1$: compresses the graph horizontally (makes it narrower)
		\item if $0 < b < 1$: stretches the graph horizontally (makes it wider)
	\end{itemize}

consider $f(x) = x^2$:\\
$g(x) = 3x^2$ (vertical scaling)\\
$h(x) = (3x)^2$ (horizontal scaling) notice how here you will be squaring the scaling factor\\

BONUS $f(x) = x$ (consider why vertical and horizontal scaling looks the same for linear)\\

$\lvert a\rvert = \lvert -a\rvert$, $\lvert ab\rvert = \lvert a\rvert\lvert b\rvert$\\
The \textbf{distance} between two real numbers $a$ and $b$ is $\lvert b - a \rvert$, which is the length of the line segment joining $a$ and $b$.\\
Two real numbers $a$ and $b$ are close to each other if $\lvert b - a\rvert$ is small, and this is the case if their decimal expansions agree to many places. More precisely, \textit{if the decimal expansions of a and b agree to k places (to the right of the decimal point), then the distance $\lvert b - a\rvert$ is at most $10^-k$. Thus, the distance between a = 3.1415 and b = 3.1478 is at most $10^{-2}$ because a and b agree to two places. In fact, the distance is exactly $\lvert3.1415 - 3.1478\rvert = 0.0063$.}\\
Beware that $\lvert a + b\rvert$ is not equal to $\lvert a\rvert + \lvert b\rvert$ unless $a$ and $b$ have the same sign or at least one of $a$ and $b$ is zero. If they have opposite sins, cancellation occurs in the sum $a + b$ and $\lvert a+b\rvert < \lvert a\rvert + \lvert b\rvert$. For example, $\lvert 2 + 5\rvert = \lvert2\rvert + \lvert5\rvert$ but $\lvert-2 + 5\rvert = 3$, which is less than $\lvert-2\rvert + \lvert5\rvert = 7$. In any case, $\lvert a + b\rvert$ is never larger than $\lvert a\rvert + \lvert b\rvert$ and this gives us the simple but important \textbf{triangle inequality}: $\lvert a + b\rvert \leq \lvert a\rvert + \lvert b\rvert$\\

$[a, b] = {x \in \mathbb{R} : a \leq x \leq b}$\\
$(a, b) = {x \in \mathbb{R} : a < x < b}$\\
$[a, b) = {x \in \mathbb{R} : a \leq x < b}$\\
$(a, b] = {x \in \mathbb{R} : a < x \leq b}$\\
$(-r, r) = {x : \lvert x\rvert < r}$\\

\textbf{composing new functions}\\
	If $f$ and $g$ are functions, we may construct new functions by forming the sum, difference, product, and quotient functions:\\
	\begin{itemize}
		\item $(f + g)(x) = f(x) + g(x)$
		\item $(f - g)(x) = f(x) - g(x)$
		\item $(fg)(x) = f(x)g(x)$
		\item $(\frac{f}{g}(x) = \frac{f(x)}{g(x)})$
	\end{itemize}
	We can also multiply functions by constants. A function of the form: $c_1f(x) + c_2g(x)$ is called a \textbf{linear combination}.\\
	\textbf{Composition} is another important way of constructing new functions. The composition of $f$ and $g$ is the function $f \circ g$ defined by $(f \circ g)(x) = f(g(x))$, defined for values of $x$ in the domain of $g$ such that $g(x)$ lies in the domain of $f$.\\
	ex. Compute the composite functions $f \circ g$ and $g \circ f$ and discuss their domains where $f(x) = \sqrt{x}$ and $g(x) = 1 - x$\\
	solution: $(f \circ g)(x) = f(g(x)) = f(1 - x) = \sqrt{1 - x}$ The square root $\sqrt{1 - x}$ is defined if $1 - x \geq 0$ or $x \leq 1$, so the domain of $f \circ g$ is ${x : x \leq 1}$.\\
	On the other hand, $(g \circ f)(x) = g(f(x)) = g(\sqrt{x}) = 1 - \sqrt{x}$ The domain of $g \circ f$ is ${x : x \geq 0}$.\\

\textbf{invertable functions}\\
	"is this function invertible?" $\Leftrightarrow$ "does an inverse function exist for this function" $\Leftrightarrow$ "is the function one-to-one?" (horizontal line test)
	\begin{itemize}
		\item if it is, then the inverse function exists
		\item if it is not, then the inverse function does not exist, and the function is not invertible (as a function)
	\end{itemize}
	consider $f(x) = x^2$ this function is not one-to-one (horizontal line test) this it is not invertable unless you restrict the domain to be $x \geq 0$.\\
	to find the inverse algeraically you can swap the x's and y's and then solve for y\\

\textbf{rational functionas}\\
$f(x) = \frac{P(x)}{Q(x)}$ where $P(x)$ and $Q(x)$ are polynomials\\
$Q(x) \ne 0$\\
the domian is all real numbers except for where the denominator is 0\\
a vertial asymptote occurs where the denominator is zero (and not canceled by a common factor)\\
holes (removable discontinuities) occur if a factor cancels from both the numerator and denominator\\
horizontal asmptotes take $n$ to be the degree of the numerator and $m$ to be the degree if the denominator
	\begin{itemize}
		\item $n < m$: horizontal asymptote at $y = 0$
		\item $n = m$: horitzontal asymptote at $\frac{\text{leading coeff. of}P(x)}{\text{leading coeff. of}Q(x)}$
		\item $n > m$: no horizontal asymptote (however there may be an oblique/slant asymptote instead)
	\end{itemize}
slant(oblique) asymptotes occur when $n = m+1$ ... use polynomial division to find slant asymptotes
the $x$ intercepts occur where the numerator is zero (where does the function = 0)\\
the $y$ intercept: plug in $x = 0$\\

\textbf{conic sections}
	\begin{itemize}
		\item ellipses
			\begin{itemize}
				\item $(h, k)$ center of the ellipse
				\item $a$ semi-major axis (long radius)
				\item $b$ semi-minor axis (short radius)
				\item $c$ distance from center to each focus $c = \sqrt{a^2 - b^2}$
			\end{itemize}
			\begin{itemize}
				\item horizontal major axis $\frac{(x - h)^2}{a^2} + \frac{(y - k)^2}{b^2} = 1$
				\item vertical major axis $\frac{(x - h)^2}{b^2} + \frac{(y - k)}{a^2} = 1$
			\end{itemize}

		\item hyperbolas
			\begin{itemize}
				\item $(h, k)$ center
				\item $a$ distance from center to each vertex (on transverse axis) 
				\item $b$ related to the asymptotes
				\item $c = \sqrt{a^2 + b^2}$ distance from center to each focus (note here add not subtract like ellipse)
				\item asymptotes
					\begin{itemize}
						\item for horizontal hyperbola $y - k = \pm\frac{b}{a}(x - h)$
						\item for vertical hyperbola $y - k = \pm\frac{a}{b}(x - h)$
					\end{itemize}
				\item transverse axis: line through both vertices and foci
				\item conjugate axis: perpendicular to the transverse axis
			\end{itemize}
			\begin{itemize}
				\item opens horizontally $\frac{(x - h)^2}{a^2} - \frac{(y - k)^2}{b^2} = 1$  
				\item opens vertically $\frac{(x - h)^2}{b^2} - \frac{(y - k)}{a^2} = 1$ 
			\end{itemize}
	\end{itemize}

\textbf{scalars and vectors}
	\begin{itemize}
		\item a scalar is a quantity that has only magnitude (size or amount) think temperature, mass, time, speed
		\item a vector has both a magnitude and a direction... think displacement, velocity, force
	\end{itemize}

	find the component form of $\vec{a}$ given:\\
	$\lvert\vec{a}\rvert = 3$ and $\theta = 30^{\circ}$\\
	$\vec{a} = (3\cos(30^{\circ}), 3\sin(30^{\circ}))$\\
	$\vec{a} = (\frac{3\sqrt{3}}{2}, \frac{3}{2})$\\

	now find the magnitude and direction of $\vec{b} = (\sqrt{2}, \sqrt{2})$\\
	$\lvert\vec{b}\rvert = \sqrt{(\sqrt{2})^2 + (\sqrt{2})^2} = 2$\\
	$\theta = \tan^{-1}(\frac{\sqrt{2}}{\sqrt{2}}) = 45^{\circ}$ (be aware of the quadrants)\\

	$\vec{w} = (1, 2)$ so $3\vec{w} = (3, 6)$\\

	$\vec{a} = (3, -1)$ and $\vec{b} = (2, 3)$\\
	$\vec{a} + \vec{b} = (3+2, -1+3) = (5, 2)$
	$\vec{a} - \vec{b} = (3-2, -1-3) = (1, -4)$ (you could also just add the negative of $\vec{b}$)\\

\textbf{complex numbers}\\
	$i = \sqrt{-1}, i^2 = -1$\\
	$z = a + bi$ rectangular form\\
	$z = r(\cos(\theta) + i\sin(\theta))$ where $r = \lvert z\rvert$ and $\theta = \tan^{-1}(\frac{b}{a})$\\

	$(2 + 3i)(1 - 4i) = 2 - 8i + 3i - 12i^2 = (14 - 5i)$\\
	$\frac{1 + i}{2 - 3i} = \frac{(1 + i)(2 + 3i)}{(2 - 3i)(2 + 3i)} = \frac{2 + 5i - 3}{4 + 9} = -\frac{1}{13} + \frac{5}{13}i$\\	
	$a^2 + b^2 = (a + bi)(a - bi)$\\

	multiplying by i has a very cool and geometric meaning in the complex plane: it corresponds to a 90-degree counterclockwise rotation about the origin.\\

	$i(a + bi) = ai + bi^2 = ai - b = -b + ai$ so $i(a + bi) = -b + ai$ that is a new complex number whose real part is -b, and imaginary part is a\\

\textbf{matrices}


\textbf{probability and combinatorics}\\


\textbf{series}\\

\section*{trigonometry}

$
\begin{array}{|c|c|c|c|}
\hline
\text{Angle (Degrees)} & \text{Angle (Radians)} & \cos(\theta) & \sin(\theta) \\
\hline
0^\circ & 0 & 1 & 0 \\
30^\circ & \frac{\pi}{6} & \frac{\sqrt{3}}{2} & \frac{1}{2} \\
45^\circ & \frac{\pi}{4} & \frac{\sqrt{2}}{2} & \frac{\sqrt{2}}{2} \\
60^\circ & \frac{\pi}{3} & \frac{1}{2} & \frac{\sqrt{3}}{2} \\
90^\circ & \frac{\pi}{2} & 0 & 1 \\
\hline
120^\circ & \frac{2\pi}{3} & -\frac{1}{2} & \frac{\sqrt{3}}{2} \\
135^\circ & \frac{3\pi}{4} & -\frac{\sqrt{2}}{2} & \frac{\sqrt{2}}{2} \\
150^\circ & \frac{5\pi}{6} & -\frac{\sqrt{3}}{2} & \frac{1}{2} \\
180^\circ & \pi & -1 & 0 \\
\hline
210^\circ & \frac{7\pi}{6} & -\frac{\sqrt{3}}{2} & -\frac{1}{2} \\
225^\circ & \frac{5\pi}{4} & -\frac{\sqrt{2}}{2} & -\frac{\sqrt{2}}{2} \\
240^\circ & \frac{4\pi}{3} & -\frac{1}{2} & -\frac{\sqrt{3}}{2} \\
270^\circ & \frac{3\pi}{2} & 0 & -1 \\
\hline
300^\circ & \frac{5\pi}{3} & \frac{1}{2} & -\frac{\sqrt{3}}{2} \\
315^\circ & \frac{7\pi}{4} & \frac{\sqrt{2}}{2} & -\frac{\sqrt{2}}{2} \\
330^\circ & \frac{11\pi}{6} & \frac{\sqrt{3}}{2} & -\frac{1}{2} \\
360^\circ & 2\pi & 1 & 0 \\
\hline
\end{array}
$\\

$1^\circ = \frac{\pi}{180}$rad\\
$1 rad = \frac{180^\circ}{pi}$\\

Trigonometric functions are special mathematical functions that originally come from studying right triangles. They relate the size of an angle in a triangle to the ratios of the lengths of the triangle's sides.\\

\textbf{SOH-CAH-TOA} is a mnemonic device that expresses the relationship between the basic trigonometric functions and the ratios of the sides in a right triangle.\\

The triangle definition only works for angles between 0 and $\frac{\pi}{2}$. To extend trig functions to all angles (including negative angles and angles larger then ($2\pi or 360^{\circ}$), mathematicians use the unit circle. Thus allowing use to use trig functions on the coordinate plane, enabling graphing and calculus.\\

To derive the rest of the fundamental trigonometric identities, you need a combination of a few key identities and principles. The most important starting point is the Pythagorean identity, but you’ll also need the basic relationships between the trigonometric functions, such as the definitions of sine, cosine, tangent, secant, cosecant, and cotangent in terms of a right triangle or the unit circle. \\

$\sin(-\theta) = -\sin(\theta)$\\
$\cos(-\theta) = \cos(\theta)$\\
$\tan(-\theta) = -\tan(\theta)$\\
$\sin(\frac{\pi}{2} - \theta) = \cos(\theta)$\\
$\cos(\frac{\pi}{2} - \theta) = \sin(\theta)$\\
$\tan(\frac{\pi}{2} - \theta) = \cot(\theta)$\\
$\sin^2 \theta + \cos^2 \theta = 1 $\\
$\sec \theta = \frac{1}{\cos \theta}, \quad \csc \theta = \frac{1}{\sin \theta}, \quad \cot \theta = \frac{1}{\tan \theta} $\\
$\tan \theta = \frac{\sin \theta}{\cos \theta}, \quad \cot \theta = \frac{\cos \theta}{\sin \theta} $\\
$1 + \tan^2 \theta = \sec^2 \theta $\\
$1 + \cot^2 \theta = \csc^2 \theta $\\
$\sin(\alpha + \beta) = \sin \alpha \cos \beta + \cos \alpha \sin \beta $\\
$\cos(\alpha + \beta) = \cos \alpha \cos \beta - \sin \alpha \sin \beta $\\
$\tan(\alpha + \beta) = \frac{\tan \alpha + \tan \beta}{1 - \tan \alpha \tan \beta} $\\
$\sin(\alpha - \beta) = \sin \alpha \cos \beta - \cos \alpha \sin \beta $\\
$\cos(\alpha - \beta) = \cos \alpha \cos \beta + \sin \alpha \sin \beta $\\
$\tan(\alpha - \beta) = \frac{\tan \alpha - \tan \beta}{1 + \tan \alpha \tan \beta} $\\
$\sin(2\theta) = 2\sin \theta \cos \theta $\\
$\cos(2\theta) = \cos^2 \theta - \sin^2 \theta = 2\cos^2 \theta - 1 = 1 - 2\sin^2 \theta $\\
$\tan(2\theta) = \frac{2\tan \theta}{1 - \tan^2 \theta} $\\
$\sin^2(2\theta) = \frac{1 - \cos^2(2\theta)}{2} $\\
$\cos^2(2\theta) = \frac{1 + \cos(2\theta)}{2} $\\
$\sin(90^\circ - \theta) = \cos \theta, \quad \cos(90^\circ - \theta) = \sin \theta $\\
$\tan(90^\circ - \theta) = \cot \theta, \quad \cot(90^\circ - \theta) = \tan \theta $\\
$\sec(90^\circ - \theta) = \csc \theta, \quad \csc(90^\circ - \theta) = \sec \theta $\\
$\sin(-\theta) = -\sin(\theta), \quad \cos(-\theta) = \cos(\theta) $\\
$\tan(-\theta) = -\tan(\theta), \quad \sec(-\theta) = \sec(\theta) $\\
$\csc(-\theta) = -\csc(\theta), \quad \cot(-\theta) = -\cot(\theta) $\\
$\sin \alpha \sin \beta = \frac{1}{2} [\cos(\alpha - \beta) - \cos(\alpha + \beta)] $\\
$\cos \alpha \cos \beta = \frac{1}{2} [\cos(\alpha - \beta) + \cos(\alpha + \beta)] $\\
$\sin \alpha \cos \beta = \frac{1}{2} [\sin(\alpha + \beta) + \sin(\alpha - \beta)] $\\

\textbf{law of sines and cosines}
	$\frac{\sin(A)}{a} = \frac{\sin(B)}{b} \frac{\sin(C)}{c}$\\
	$a^2 = b^2 + c^2 - 2bc\cos(A)$\\

\textbf{sinusoidal equations}

\section*{limits}
	so let us consider $f(x) = \frac{\sin(x)}{x}$ ($x$ in radians)\\
	the value of $f(0)$ is not defined because\\
	$f(0) = \frac{\sin(0)}{0} = \frac{0}{0}$\\
	however if we take $x \to 0^{+}$ and $x \to 0^{-}$ we get the impression that $f(x)$ gets closer and closer to 1 as $x \to 0$ from the left and right.
	\begin{center}
		$\lim_{x \to 0}f(x) = 1$
	\end{center}
	recall that the distance between two numbers $\lvert a - b\rvert$...thus we can express the idea that $f(x)$ is close to $L$ by saying that $\lvert f(x) - L\rvert$ is small.\\

	\textbf{informal limit definition} Assume that $f(x)$ is defined for all $x$ near $c$ (i.e., in some open interval containing c), but not necessarily at $c$ itself. We say that:\\
	\textit{the limit of $f(x)$ as $x$ approaches $c$ is equal to $L$}\\
	if $\lvert f(x) - L\rvert$ becomes arbitarily small when $x$ is any number sufficiently close (but not equal) to $c$. In this case, we write:
	\begin{center}
		$\lim_{x \to c}f(x) = L$
	\end{center}
	we also say the $f(x)$ approaches or converges to $L$ as $x \to c$ (and we write $f(x) \to L$)\\

	If the values of $f(x)$ do not converge to any limit as $x \to c$, we say that $\lim_{x \to c}f(x)$ DNE. It is important to note that the value $f(c)$ itself, which may or may not be defined, plays no role in the limit. All that matters are the values $f(x)$ for $x$ close to c. Furthermore, if $f(x)$ approaches a limit as $x \to c$, then the limiting value $L$ is unique.\\

	We say the $\lim_{x \to c}f(x) = \infty$ if $f(x)$ increases beyond bound as $x$ approaches $c$, and $\lim_{x \to c}f(x) = -\infty$ if $f(x)$ becomes arbitrarily large (in absolute value) but negative as $x$ approaches $c$

\section*{differentiation}
$\frac{d}{dx}(c) = 0$\\
$\frac{d}{dx}x = 1$\\
$\frac{d}{dx}(x^n) = nx^{-1}$ (power rule)\\
$\frac{d}{dx}[cf(x)] = cf'(x)$\\
$\frac{d}{dx}[f(x)+g(x)] = f'(x) + g'(x)$\\
$\frac{d}{dx}[f(x)g(x)] = f(x)g'(x) + g(x)f'(x)$ (product rule)\\
$\frac{d}{dx}[\frac{f(x)}{g(x)}] = \frac{g(x)f'(x) - f(x)g'(x)}{[g(x)]^2}$ (quotient rule)\\
$\frac{d}{dx}f(g(x)) = f'(g(x))g'(x)$ (chain rule)\\
$\frac{d}{dx}f(x)^n = nf(x)^{n-1}f'(x)$ (general power rule)\\
$\frac{d}{dx}\sin(x) = \cos(x)$\\ 
$\frac{d}{dx}\cos(x) = -\sin(x)$\\
$\frac{d}{dx}\tan(x) = \sec^2(x)$
$\frac{d}{dx}\csc(x) = -\csc(x)\cot(x)$\\
$\frac{d}{dx}\sec(x) = \sec(x)\tan(x)$\\
$\frac{d}{dx}\cot(x) = -\csc^2(x)$\\
$\frac{d}{dx}\sin^{-1}(x) = \frac{1}{\sqrt(1 - x^2)}$\\
$\frac{d}{dx}\cos^{-1}(x) = -\frac{1}{\sqrt(1 - x^2)}$\\
$\frac{d}{dx}\tan^{-1}(x) = \frac{1}{1 + x^2}$\\
$\frac{d}{dx}(e^x) = e^x$\\
$\frac{d}{dx}(a^x) = (\ln a)a^x$\\
$\frac{d}{dx}\ln\mid x\mid = \frac{1}{x}$\\
$\frac{d}{dx}\log_ax = \frac{1}{(\ln a)x}$\\

\section*{integration}



\section*{essential theorems}
	\begin{itemize}
		\item the fundamental theorem of algebra
		\item squeeze theorem
		\item intermediate value theorem
		\item mean value theorem
		\item extreme value theorem
		\item fundamental theorem of calculus part I and II
	\end{itemize}

\newpage
\section*{review problems}




\end{document}
