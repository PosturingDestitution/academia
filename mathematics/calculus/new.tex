\documentclass{article}
\usepackage{amsmath}
\usepackage{amssymb}
\usepackage{xcolor}
\setlength{\parindent}{0pt}

\title{integration and more}
\author{alexander}
\date{\today}

\begin{document}
\maketitle

\textbf{trigonometric functions and the inverse}\\
to obtain the inverse of a function, the function must be one-to-one, meaning that each input corresponds to exactly one unique output and no two different inputs share the same output value. however, the trigonometric functions such as sine, cosine, and tangent are not one-to-one over their entire domains because they are periodic and repeat their values infinitely many times. therefore, to define their inverses $\arcsin(x)$, $\arccos(x)$, and $\arctan(x)$ we must restrict the domain of each trigonometric function to an interval where it passes the horizontal line test. this ensures that each inverse function is well-defined and produces a single, unique ouput for every input within its range.	
	\begin{itemize}
		\item $y = \sin(x)$, domain: $x \in (-\infty, \infty)$, range: $y \in [-1, 1]$, period: $2\pi$\\
			$y = \arcsin(x)$, domain: $x \in [-1, 1]$, range: $y = [-\frac{\pi}{2}, \frac{\pi}{2}$
		\item $y = \cos(x)$, domain: $x \in (-\infty, \infty)$, range: $y \in [-1, 1]$, period: $2\pi$\\
			$y = \arccos(x)$, domain: $x \in [-1, 1]$, range: $y \in [0, \pi]$
		\item $y = \tan(x) = \frac{\sin(x)}{\cos(x)}$, domain: $x \neq \frac{\pi}{2}, n \in \mathbb{Z}$, range: $y \in (\infty, \infty)$, period: $\pi$\\
			$y = \arctan(x)$, domain: $x \in (-\infty, \infty)$, range: $\in (-\frac{\pi}{2}, \frac{\pi}{2})$
	\end{itemize}	

\textbf{FTC}
	\begin{itemize}
		\item if $f$ is a continuous function on an interval $[a, b]$, and $F(x) = \int_{a}^{x}f(t)\,dt$ is defined for $x \in [a, b]$, then $F'(x) = f(x)$
		\item if $f$ is continuous function on $[a, b]$, and $F$ is any andiderivative of $f$, meaning $F'(x) = f(x)$, then $\int_{a}^{b}f(x)\,dx = F(b) - F(a)$
	\end{itemize}

\textbf{integration power rule}
        \begin{itemize}
                \item $\frac{d}{dx}[\frac{x^{n+1}}{n+1}] = x^n \Rightarrow \int x^n\,dx = \frac{x^{n+1}}{n+1} + C$
        \end{itemize}
                
\textbf{u-substitution}
        \begin{itemize}
                \item $\int f(g(x))g'(x)\,dx \Rightarrow \int f(u)\,du$
		\item $\int_{a}^{b}f(g(x))g'(x)\,dx = \int_{u(a)}^{u(b)}f(u)\,du$
        \end{itemize}
                
\textbf{integration by parts}
        \begin{itemize}
                \item $(uv)'(x) = u'(x)v(x) + u(x)v'(x) \Rightarrow \int u\,dv = uv - \int v\,du$
        \end{itemize}

\textbf{trigonometric substitution}
	\begin{enumerate}
		\item $\sqrt{a^2 - b^2x^2} = \sqrt{a^2(1 - \frac{b^2x^2}{a^2})} = \sqrt{a^2(1 - (\frac{bx}{a})^2)} = \sqrt{a^2(1 - (\frac{b}{a}x)^2)} = a\sqrt{1 - (\frac{b}{a}x)^2}$
		\item $\sin^2(\theta) = \cos^2(\theta) = 1 \Leftrightarrow \cos(\theta) = \sqrt{1 - (\sin(\theta))^2}$ 
		\item $\frac{b}{a}x = \sin(\theta) \Leftrightarrow x = \frac{a}{b}\sin(\theta)$
		\item $a\cos(\theta) = \sqrt{a^2 - b^2x^2}$
	\end{enumerate}

when you do a trig substitution, you often let something like $x = \sin(\theta)$. that automatically restrics $\theta$ to the range of the inverse sine function, which is $-\frac{\pi}{2} \leq \theta \leq \frac{\pi}{2}$. within this range, $\sin(\theta)$ can be positive or negative, but $\cos(\theta)$ is always nonnegative. now, when you use the pythagorean identity $\sin^2(\theta) + \cos^2(\theta) = 1$, and solve for $\cos(\theta)$, you get $\cos(\theta) = \pm \sqrt{1 - \sin^2(\theta)}$. mathematically, both the positive and negative square roots are valid - but you have to choose which sign is correct for the range of $\theta$ you are working in. because we restriced $\theta$ to $[-\frac{\pi}{2}, \frac{\pi}{2}]$ when we said $x = \sin(\theta)$, we know $\cos(\theta) \geq 0$ there. Therefore, we choose the positive square root: $\cos(\theta) = + \sqrt{1 - \sin^2(\theta)}$. if instead we had chosen a substitution involving $\cos(\theta)$, we would pick a domain where $\sin(\theta)$ has a definite sign and make the corresponding choice.\\

\textbf{integration by partial fraction decomposition}
	\begin{itemize}
		\item distinct linear factors $\frac{A}{x - a} + \frac{B}{x - b}$
		\item repeated linear factors $\frac{A_1}{x - a} + \frac{A_2}{(x - a)^2} + \ldots + \frac{A_n}{(x - a)^n}$
		\item irreducible quadratic factors $\frac{Ax + B}{x^2 + bx + c}$
		\item repeated quadratic factors $\frac{A_1x + B_1}{x^2 + bx + c} + \ldots + \frac{A_nx + B_n}{(x^2 + bx + c)^n}$
	\end{itemize}

a rational function is of the form $\frac{P(x)}{Q(x)}$ where $P(x)$ and $Q(x)$ are polynomials. with a proper fraction ($\text{deg}(P) < \text{deg}(Q)$) you can go straight to the decomposition, but for something improper ($\text{deg}(P) > \text{deg}(Q)$), you will need to use polynomial division ($\frac{P(x)}{Q(x)} = S(x) + \frac{R(x)}{Q(x)}$) where $S(x)$ is a polynomial (the quotient), and $\frac{R(x)}{Q(x)}$ is now a proper fraction (remainder over the denominator).

\textbf{riemann sums}
	\begin{itemize}
		\item given a definite integral like so: $\int_{a}^{b}f(x)\,dx$ you can approximate it by breaking $[a, b]$ into smaller subintervals with a width of $\Delta x = \frac{b - a}{n}$. you now have subintervals like so: $[x_0, x_1], [x_1, x_2], \ldots, [x_{n-1}, x_n]$ where $x_i = a + i\Delta x$.
		\item $L_n = \sum_{i = 0}^{n - 1}f(x_i) \cdot \Delta x$ 
		\item $R_n = \sum_{i = 1}^{n}f(x_i) \cdot \Delta x$
		\item $M_n = \sum_{i = 1}^{n}f(m_i) \cdot \Delta x$ where $m_i = \frac{x_{i-1} + x_i}{2}$
		\item $T_n = \sum_{i = 1}^{n}\frac{f(x_{i - 1} + f(x_i))}{2} \cdot \Delta x$
	\end{itemize}

\textbf{combinatorics}
if you have n objects you can arrange them in n! ways. lets say n is 5 then using the multiplication rule you can place 5 unique objects in the first location leaving you with 4 unique object for the next location and so on. now lets say you have 5 objects again and want to find the number of ways to arrange (permute) 3 of those 5 elements ($P(n, k)$) so 123 is a different arrangment from 321 which means that the order matters. this being said $P(n, k) = \frac{n!}{(n - k)!} \Rightarrow P(5, 3) = \frac{5 * 4 * 3 * 2 * 1}{2 * 1}$. 5 objects can go in the first bucket, 4 can go in the next and finally 3 remain for the last bucket. now for a combination we say that the order does not matter meaning that 123 = 321 for this case. the number of ways you can arrange 3 objects would be 3! so we will have to remove those from the calculation like so $\binom{n}{k} = \frac{n!}{k!(n-k)!} \Rightarrow \binom{5}{3} = \frac{5!}{3!(5 - 3)!}$\\

combinations again\\

pascals identity: $\binom{n}{k} = \binom{n-1}{k-1} + \binom{n-1}{k}$\\

prove/understand the binomial theorem\\







signed areas\\
volumes of revolution\\
arclength\\


\textbf{logs, e, and more}\\
in the 1600s, people invented logarithms to make multiplication and division easier. at the time, atronomers and navigators were doing tons of tedious multiplications. the insight was that multiplication can be turned into addition if you can map numbers into a new scale $\log(ab) = \log(a) + \log(b)$ so instead of multiplying $a$ and $b$, you just add their logs. before calculators, people used log tables. the slide rule was used as a physical calculator (with log scales) up until the 1960s (cold war).\\

compound interest formula\\
$\lim_{n \to \infty}$\\
$\frac{dy}{dx} = y$ show that $e^xe^X = e^{x + X}$\\
factorial\\
geometric series\\
$\frac{dy}{dx} = cy \Rightarrow y(x) = e^{cx}$\\
doubling time and half life\\
lhopitals rule\\
hyperbolic functions and a "leap of magination"\\


numerical integration\\
improper integrals\\

taylor polynomials\\

intro to first order linear diffyQ\\
infinite series\\
\end{document}
