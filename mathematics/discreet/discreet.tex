\documentclass{article}
\usepackage[utf8]{inputenc}

\title{discreet}
\author{alexander}

\begin{document}
\maketitle

\section{permutations}
The permutation formula is used to find the number of ways to arrange $r$ items from a set of $n$ items, where order matters 
\begin{center} $nPr = \frac{n!}{(n-r)!}$ \end{center}
	\begin{itemize}
		\item $n$ is the total number of items in the set
		\item $r$ is the number of items being chosen from the set
	\end{itemize}
If we want to choose $r$ items, we need to exclude the last $(n-r)$ items from the factorial. That's why we divide by $(n-r)!$

\section{combinations}
Combinations are used to find the number of ways to choose $r$ items from a set of $n$, where order does not matter
\begin{center} $nCr = \frac{n!}{r!(n-r)!}$ \end{center}
	\begin{itemize}
		\item $n$ is the total number of items in the set
		\item $r$ is the number of items being chosen from the set
	\end{itemize}

\section{terms}
	\begin{itemize}
		\item repetition: when an object can be chosen more than once. this is often indicated by allowing duplicate items within a selection
		\item order (arrangement): order matters in combinatorial problems where the arrangement of selected objects influences the outcome
		\item replacement: when the choice or selection does not exclude the possibility of picking an item more than once from the original pool
	\end{itemize}

\section{misc.}
	\begin{itemize}
		\item binary decision diagram and trees:
		\item pigeonhole principle:
		\item stirling numbers:
		\item bell numbers:
		\item multinomial coefficients:
		\item permutations with fixed points and cycles:
		\item lattice theory:
		\item generation functions:
		\item recurrence relations:
	\end{itemize}

\end{document}























