\documentclass{article}
\usepackage{amsmath}
\usepackage{amssymb}
\usepackage{xcolor}
\setlength{\parindent}{0pt}

\title{integration and more}
\author{alexander}
\date{\today}

\begin{document}
\maketitle

\textbf{trigonometric functions and the inverse}\\
to obtain the inverse of a function, the function must be one-to-one, meaning that each input corresponds to exactly one unique output and no two different inputs share the same output value. however, the trigonometric functions such as sine, cosine, and tangent are not one-to-one over their entire domains because they are periodic and repeat their values infinitely many times. therefore, to define their inverses $\arcsin(x)$, $\arccos(x)$, and $\arctan(x)$ we must restrict the domain of each trigonometric function to an interval where it passes the horizontal line test. this ensures that each inverse function is well-defined and produces a single, unique ouput for every input within its range.	
	\begin{itemize}
		\item $y = \sin(x)$, domain: $x \in (-\infty, \infty)$, range: $y \in [-1, 1]$, period: $2\pi$\\
			$y = \arcsin(x)$, domain: $x \in [-1, 1]$, range: $y = [-\frac{\pi}{2}, \frac{\pi}{2}$
		\item $y = \cos(x)$, domain: $x \in (-\infty, \infty)$, range: $y \in [-1, 1]$, period: $2\pi$\\
			$y = \arccos(x)$, domain: $x \in [-1, 1]$, range: $y \in [0, \pi]$
		\item $y = \tan(x) = \frac{\sin(x)}{\cos(x)}$, domain: $x \neq \frac{\pi}{2}, n \in \mathbb{Z}$, range: $y \in (\infty, \infty)$, period: $\pi$\\
			$y = \arctan(x)$, domain: $x \in (-\infty, \infty)$, range: $\in (-\frac{\pi}{2}, \frac{\pi}{2})$
	\end{itemize}	

\textbf{FTC}
	\begin{itemize}
		\item if $f$ is a continuous function on an interval $[a, b]$, and $F(x) = \int_{a}^{x}f(t)\,dt$ is defined for $x \in [a, b]$, then $F'(x) = f(x)$
		\item if $f$ is continuous function on $[a, b]$, and $F$ is any andiderivative of $f$, meaning $F'(x) = f(x)$, then $\int_{a}^{b}f(x)\,dx = F(b) - F(a)$
	\end{itemize}

\textbf{integration power rule}
        \begin{itemize}
                \item $\frac{d}{dx}[\frac{x^{n+1}}{n+1}] = x^n \Rightarrow \int x^n\,dx = \frac{x^{n+1}}{n+1} + C$
        \end{itemize}
                
\textbf{u-substitution}
        \begin{itemize}
                \item $\int f(g(x))g'(x)\,dx \Rightarrow \int f(u)\,du$
		\item $\int_{a}^{b}f(g(x))g'(x)\,dx = \int_{u(a)}^{u(b)}f(u)\,du$
        \end{itemize}
                
\textbf{integration by parts}
        \begin{itemize}
                \item $(uv)'(x) = u'(x)v(x) + u(x)v'(x) \Rightarrow \int u\,dv = uv - \int v\,du$
        \end{itemize}

\textbf{trigonometric substitution}
	\begin{enumerate}
		\item $\sqrt{a^2 - b^2x^2} = \sqrt{a^2(1 - \frac{b^2x^2}{a^2})} = \sqrt{a^2(1 - (\frac{bx}{a})^2)} = \sqrt{a^2(1 - (\frac{b}{a}x)^2)} = a\sqrt{1 - (\frac{b}{a}x)^2}$
		\item $\sin^2(\theta) = \cos^2(\theta) = 1 \Leftrightarrow \cos(\theta) = \sqrt{1 - (\sin(\theta))^2}$ 
		\item $\frac{b}{a}x = \sin(\theta) \Leftrightarrow x = \frac{a}{b}\sin(\theta)$
		\item $a\cos(\theta) = \sqrt{a^2 - b^2x^2}$
	\end{enumerate}

when you do a trig substitution, you often let something like $x = \sin(\theta)$. that automatically restrics $\theta$ to the range of the inverse sine function, which is $-\frac{\pi}{2} \leq \theta \leq \frac{\pi}{2}$. within this range, $\sin(\theta)$ can be positive or negative, but $\cos(\theta)$ is always nonnegative. now, when you use the pythagorean identity $\sin^2(\theta) + \cos^2(\theta) = 1$, and solve for $\cos(\theta)$, you get $\cos(\theta) = \pm \sqrt{1 - \sin^2(\theta)}$. mathematically, both the positive and negative square roots are valid - but you have to choose which sign is correct for the range of $\theta$ you are working in. because we restriced $\theta$ to $[-\frac{\pi}{2}, \frac{\pi}{2}]$ when we said $x = \sin(\theta)$, we know $\cos(\theta) \geq 0$ there. Therefore, we choose the positive square root: $\cos(\theta) = + \sqrt{1 - \sin^2(\theta)}$. if instead we had chosen a substitution involving $\cos(\theta)$, we would pick a domain where $\sin(\theta)$ has a definite sign and make the corresponding choice.\\

\textbf{integration by partial fraction decomposition}
	\begin{itemize}
		\item distinct linear factors $\frac{A}{x - a} + \frac{B}{x - b}$
		\item repeated linear factors $\frac{A_1}{x - a} + \frac{A_2}{(x - a)^2} + \ldots + \frac{A_n}{(x - a)^n}$
		\item irreducible quadratic factors $\frac{Ax + B}{x^2 + bx + c}$
		\item repeated quadratic factors $\frac{A_1x + B_1}{x^2 + bx + c} + \ldots + \frac{A_nx + B_n}{(x^2 + bx + c)^n}$
	\end{itemize}

a rational function is of the form $\frac{P(x)}{Q(x)}$ where $P(x)$ and $Q(x)$ are polynomials. with a proper fraction ($\text{deg}(P) < \text{deg}(Q)$) you can go straight to the decomposition, but for something improper ($\text{deg}(P) > \text{deg}(Q)$), you will need to use polynomial division ($\frac{P(x)}{Q(x)} = S(x) + \frac{R(x)}{Q(x)}$) where $S(x)$ is a polynomial (the quotient), and $\frac{R(x)}{Q(x)}$ is now a proper fraction (remainder over the denominator).

\textbf{riemann sums}
	\begin{itemize}
		\item given a definite integral like so: $\int_{a}^{b}f(x)\,dx$ you can approximate it by breaking $[a, b]$ into smaller subintervals with a width of $\Delta x = \frac{b - a}{n}$. you now have subintervals like so: $[x_0, x_1], [x_1, x_2], \ldots, [x_{n-1}, x_n]$ where $x_i = a + i\Delta x$.
		\item $L_n = \sum_{i = 0}^{n - 1}f(x_i) \cdot \Delta x$ 
		\item $R_n = \sum_{i = 1}^{n}f(x_i) \cdot \Delta x$
		\item $M_n = \sum_{i = 1}^{n}f(m_i) \cdot \Delta x$ where $m_i = \frac{x_{i-1} + x_i}{2}$
		\item $T_n = \sum_{i = 1}^{n}\frac{f(x_{i - 1} + f(x_i))}{2} \cdot \Delta x$
	\end{itemize}

\section*{review/rewriting (always ask why?)}
\begin{enumerate}
	\item signed area
	\item areas between two curves
	\item change of variables
	\item volume as the integral of cross-sectional area
	\item average value of a function
	\item mean value theorem for integrals
	\item volume of a solid of revolution: disk method, shell method
	\item arc lengths
	\item review properties of logs and exponents
	\item defining the number e
	\item derivative of the inverse
	\item derivative of $\ln(x)$ and the antiderivative of $\frac{1}{x}$
	\item compound interest, $P(t) = P_0e^{kt}$, doubling/halving time, defining e as a limit vs a sum
	\item L'Hopital's rule
	\item complex numbers, hyperbolic trig functions
	\item numeric integration techniques, simpsons rule, bounds of error
	\item improper integrals
	\item taylor polynomials
	\item (some) differential equations
	\item sequences, convergence, power series, taylor series
	\item polar coords
\end{enumerate}
\end{document}
