\documentclass{article}

\title{stats101}
\author{alexander}

\begin{document}
\setlength{\parindent}{0pt}
\maketitle

Statistics is the study of how to gather, summarize, and draw conclusions from data, often in presence of uncertainty.\\

A key distinction in statistics is between inference (making conclusions about a larger population) and 
description (summarizing and describing data). Inferential statistics involves making educated guesses 
based on sample data, while descriptive statistics provide summaries and descriptions of data without any 
attempt to make inferences.\\

data: a collection of values, characteristics, or observations that can be analyzed and interpreted.\\
raw data: unprocessed and unstructured data collected from various sources.\\

types:
	\begin{itemize}
		\item quantitative: numerical data that can be measured, such as heights or weights
		\item qualitative: non-numerical data that cannot be measured, such as text or categorical labels
		\item categorical: data organized into categories or groups, such as colors or brands
	\end{itemize}

characteristics:
	\begin{itemize}
		\item scalability: the ability to increase or decrease the size of the dataset without affecting its structure
		\item completeness: the presence or absence of all relevant data points in the dataset
		\item consistency: the uniformity and accuracy of the data within the dataset
	\end{itemize}

sources:
	\begin{itemize}
		\item primary: original data collected directly from sources, such as surveys or experiments
		\item secondary: existing data that is reused or repurposed, such as publicly available datasets or literature reviews
		\item external: data sourced from external organizations or systems, such as social media platforms or government databases
	\end{itemize}

quality:
	\begin{itemize}
		\item validity: the degree to which the data accurately represents the real world
		\item reliability: the consistency and accuracy of the data over time
		\item accuracy: the closeness of a measured value to its true value
	\end{itemize}

Central tendency is a statistical measure that describes the "center" or typical value of a dataset. It provides a single value that best represents the middle or avaerage of a set of data.
measures of central tendency:
	\begin{itemize}
		\item mean: the average value of a dataset
		\item median: the middle value of a dataset when sorted in ascending or descending order
		\item mode: the most frequiently occurring value in a dataset (if there is one)
	\end{itemize}

Variability refers to the amount of scatter or dispersion present in a dataset. It measures how much individual data poiunt deviate from the central tendency. In other words, it describes the spread or range of values within the dataset.
measures of variability:
	\begin{itemize}
		\item range: the difference between the largest and smallest values in a dataset
		\item variance: the average of the squared differences from the mean
		\item standard deviation: a measure of the spread or dispersion of a dataset
		\item skewness: a measure of the asymmetry of the distribution, with positive skew indicating a longer tail to the right and negative skew indicating a longer tail to the left.
		\item kurtosis: a measure of the "tailedness" or "peakedness" of the distribution
	\end{itemize}

inferences:
	\begin{itemize}
		\item descriptive inference: making statements about the caracteristics of a population based on a sample, such as estimating a population mean or proportion
		\item analytical inference: drawing conclusions about the relationships between variables within a population
	\end{itemize}

procedures:
	\begin{itemize}
		\item hypothesis testing: formulating hypotheses and testing them against data to determine if they are supported or rejected
		\item confidence intervals: constructing intervals around sample estimates that have a specific level of confidence for containing the true population parameter
		\item regression analysis: modeling relationships between variables within a population
	\end{itemize}

assumptions:
	\begin{itemize}
		\item independence: observations are independent are not influenced by each other
		\item normality: the distribution of residuals or errors follows a normal distribution
		\item homoscedasticity: the variance of residuals is constant across all levels of predictor variables
	\end{itemize}

limitations:
	\begin{itemize}
		\item sampling error: errors that occur due to the fact that only a sample is used instead of the entire population
		\item biased sampling: samples are not representative of the population or have some inherent flaws.
		\item limited generalizability: inferences may be restricted to the specific context and conditions of the study.
	\end{itemize}




\end{document}
