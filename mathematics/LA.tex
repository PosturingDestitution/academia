\documentclass{article}

\usepackage{times}
\usepackage[left=1in,right=1in,top=1in,bottom=1in]{geometry}
\usepackage{enumitem}
\usepackage{amsmath}
\usepackage{amssymb}
\usepackage{tikz}
\usepackage{xcolor}

\title{Linear Algebra}
\author{Alexander}
\date{\today}

\begin{document}
\maketitle

NOTE: For these notes, I will be using \textit{Linear Algrebra with Applications Seventh Edition} by Gareth Williams and the internet.

\section{Linear Equation, Vectors, and Matrices}

An equation in the variables $x$ and $y$ that can be written in the form $ax + by = c$, where $a$, $b$, and $c$ are real constants ($a$ and $b$ not both zero), is called a linear equation. The graph of such an equation is a straight line in the xy plane. Consider the system of two linear equations,
\begin{center}
\begin{align*}
x + y & = 5\\
2x - y &= 4
\end{align*}
\end{center}
A pair of values of $x$ and $y$ that satisfies both equations is called a solution. It can be seen by substitution that $x = 3$, $y = 2$ is a solution to this system. A solution to such a system will be a point at which the graphs of the two equations intersect. The following examples illustrate that three possibilities can arise for such systems of equations. There can be a unique solution, no solution, or many solutions. We use the point/slope form $y = mx + b$, where $m$ is the slope and $b$ is the y-intercept, to graph these lines.

Unique solution: these have different slopes
\begin{align*}
x + y &= 5 \\
2x - y &= 4
\end{align*}

No solution: these have the same slopes and different y-intercepts
\begin{align*}
-2x + y &= 3\\
-4x + 2y &= 2
\end{align*}

Many solutions: these have the same slope and the same y-intercepts
\begin{align*}
4x - 2y = 6\\
6x -3y = 9
\end{align*}

You can think of the number of equations in a system as analogous to the number of constraints or objects you have in a geometric space. Each equation typically represents a constraint that reduces the dimensionality of the solution space.\\

1 Variable: An equation in one variable (e.g., $x=a$) represents a point on a number line, which is zero-dimensional.\\
2 Variables: An equation in two variables (e.g., $y=mx+b$) represents a line in a two-dimensional space, which is one-dimensional.\\
3 Variables: An equation in three variables (e.g., $z=ax+by+c$) represents a plane in three-dimensional space, which is two-dimensional.\\
n Variables: More generally, an equation involving n variables defines an n-1 dimensional hyperplane in n-dimensional space.\\

1. Understanding Dimensions and Equations\\
Dimensions: The dimensionality of a space refers to the number of coordinates needed to describe a point within that space. For example:
\begin{itemize}
\item A point in 0 dimensions (0D).
\item A line in 1 dimension (1D), which can be described by one variable (like x).
\item A plane in 2 dimensions (2D), described by two variables (like x and y).
\item A volume in 3 dimensions (3D), described by three variables (like x, y, and z).
\end{itemize}

2. Equations as Constraints\\
When you introduce equations, they impose constraints on the variables:
\begin{itemize}
\item One Equation:
	\begin{itemize}
	\item In 2D, an equation like $y = mx + b$ represents a line. This line is one-dimensional, meaning you can move along it using a single parameter (for example, x).
	\end{itemize}
\item Two Equations:
	\begin{itemize}
	\item If you have two equations, such as:
	\begin{align*}
	y &= mx + b_1\\
	y &= mx + b_2\\
	\end{align*}
	In 2D, these represent two lines, If they intersect, they define a unique solution-a single point, which is zero-dimensional
	\end{itemize}
\end{itemize}

3. Generalizing to Higher Dimensions\\
Three Equations:\\
\begin{itemize}
\item In 3D, three equations can define a point or a line, dpending on how they intersect. For instance:
\begin{align*}
z &= ax + by + c\\
z &= dx + ey + f\\
z &= gx + hy + i\\
\end{align*}
\end{itemize}
If these planes intersect at a single point, the solution is zero-dimensional (a specific point)\\

4. Conclusion
\begin{itemize}
\item The number of equations can be thought of as the number of constraints or "objects" you are using to limit the solution space.
\item Generally, for $n$ variables, if you have $k$ independent equations, you can expect the solution space to be $n-k$ dimensions:
	\begin{itemize}
	\item if $k = n$, you have a unique solution (0D).
	\item if $k < n$, the solution space remains $n - k$ dimensional, meaning you have more freedom in choosing solutions.
	\end{itemize}
\end{itemize}

Our aim in this chapter is to analyze larger systems of linear equations. A linear equation in $n$ variables $x_1$, $x_2$, $x_3$, $\dots$, $x_n$ is one that can be written in the form
\begin{center}
$a_1x_1 + a_2x_2 + a_3x_3 + \dots + a_nx_n = b$
\end{center}
where the coefficients $a_1$, $a_2$, $\dots$, $a_n$ and $b$ are constants. The following is an example of a system of three linear equations.
\begin{center}
\begin{align*}
x_1 + x_2 + x_3 &= 2\\
2x_1 + 3x_2 + x_3 &= 3\\
x_1 + x_2 + 2x_3 &= -6
\end{align*}
\end{center}


\end{document}