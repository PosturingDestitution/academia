\documentclass{article}

\title{Operating Systems: Internals and Design Principles (Eighth Edition) chapter review}
\date{\today}
\begin{document}

\maketitle

"the book is divided into five parts: background, processes, memory, scheduling, advanced topics (embedded OSs, virtual machines, OS security, and distributed systems)"\\

"Many treatments of operating systems bunch all of the material on processes at the beginning and then deal with other topics. This is certainly valid. However, the central significance of memory management, which I believe is of equal importance to process management, has led to a decision to present this material prior to an in-depth look at scheduling. The ideal soution is for the student, after completeing Chapters 1 through 3 in series, to read and absorb the following chapters in parallel: 4 followed by (optional) 5; 6 followed by 7; 8 followed by (optional) 9; 10. The remaining parts can be done in any order. However, although the human brain may engage in parallel processing, the human student finds it impossible (and expensive) to work sucessfully with four copies of the same book simultaneously open to four different chapters. Given the necessity for linear ordering, I think that the ordering used in this book is the most effective. A final comment. Chapter 2, especially Section 2.3, provides a top-level view of all the key concepts covered in later chapters. Thus after reading Chapter 2, there is considerable flexibility in choosing the order in which to read the remaining chapters."

\section*{CHAPTER 1}
	Learning Objectives:
	\begin{itemize}
		\item Describe the basic elements of a computer system and their interrelationship.
		\item Explain the steps taken by a processor to execute an instruction.
		\item Understand the concept of interrupts and how and why a processor uses interrupts.
		\item List and describe the levels of a typical computer memory hierarchy.
		\item Explain the basic characteristics of multiprocessor and multicore organizations.
		\item Discuss the concept of locality and analyze the performance of a multilevel memory hierarchy.
		\item Understand the operation of a stack and its use to support procedure call and return.
	\end{itemize}

\section*{CHAPTER 2}
	Learning Objectives:
	\begin{itemize}
		\item Summarize, at a top level, the key functions of an operating system (OS).
		\item Discuss the evolution of operating systems for early simple batch systems to modern complex systems.
		\item Give a brief explanation of each of the major achievements in OS research, as defined in Section 2.3.
		\item Discuss the key design areas that have been instrumental in the development of modern operating systems.
		\item Define and discuss virtual machines and virtualization.
		\item Understand the OS design issues raised by the introduction of multiprocessor and multicore organization.
		\item Understand the basic structure of Windows 7.
		\item Describe the essential elements of a traditional UNIX system.
		\item Explain the new features found in modern UNIX systems.
		\item Discuss Linux and its relationship to UNIX.
	\end{itemize}




\end{document}
