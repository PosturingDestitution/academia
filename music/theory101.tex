\documentclass{article}

\title{Music Theory 101}
\author{alexander}
\date{\today}

\setlength{\parindent}{0pt}

\begin{document}
\maketitle

the circle fifths can give you insight into several things: key signatures, chord relationship, modulation, relative minors, harmonic progessions. the circle of fifths is extremely useful, but music theory has other layers: scales and modes, intervals, chord constuction, rhythm and meter, voice leading and counterpoint. think of the circle of fifths as a map of key relationships - it will not teach you everything about melody, rhythm, or chord tensions, but it is one of the most efficient "shortcuts" for understanding harmony.\\

the harmonic series, a natual phenomenon that emerges from vibrating objects. when a string, air column, or other resonant body vibrates, it does not just vibrate at a single frequency - it produces a series of overtones at integer multiples of the fundamental frequency. for example, if a string vibrates at 100Hz, it also vibrates at 200Hz, 300Hz, 400Hz, and so on. these overtones are not random; they explain why certain intervals, like octaves, fifths, and fouths, sound consonant to us. the harmonic series provides the physical basis for our sense of pitch, for the structure of scales, and for the construction of chords, liking the 12-tone system and diatonic scales back to physics and the natural laws of sound.


\end{document}
