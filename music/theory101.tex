\documentclass{article}

\title{Music Theory 101}
\author{alexander}
\date{\today}

\setlength{\parindent}{0pt}

\begin{document}
\maketitle

Pitch is how we perceive frequency.\\

down $\leftarrow$ $\rightarrow$ up\\
\textbf{C, C\#, D, D\#, E, F, F\#, G, G\#, A, A\#, B}\\
\textbf{C, Db, D, Eb, E, F, Gb, G, Ab, A, Bb, B}\\

\section*{intervals}

A semitone is the smallest interval in Western music - it is the distance between two adjacent notes on a piano. A tone (or whole step) is made up of two semitones.

\begin{table}[h!]
\centering
\begin{tabular}{|l|c|}
\hline
\textbf{Interval Name} & \textbf{Semitones} \\
\hline
Minor 2nd       & 1  \\
Major 2nd       & 2  \\
Minor 3rd       & 3  \\
Major 3rd       & 4  \\
Perfect 4th     & 5  \\
Tritone         & 6  \\
Perfect 5th     & 7  \\
Minor 6th       & 8  \\
Major 6th       & 9  \\
Minor 7th       & 10 \\
Major 7th       & 11 \\
Perfect Octave  & 12 \\
\hline
\end{tabular}
\end{table}

\section*{scales, keys, key signatures, modes, chords}

A scale is a set of musical notes ordered by pitch, which serves as the building block for melodies, harmonies, and keys. Think of it as a musical palette from which composers and performers choose notes.\\

some scales:
	\begin{itemize}
		\item major (1 - 2 - 3 - 4 - 5 - 6 - 7)
		\item natrual minor (1 - 2 - b3 - 4 - 5 - b6 - b7)
		\item harmonic minor (natual minor w/ raised 7th)
		\item melodic minor (raises 6th and 7th on ascending, natrual minor descending)
		\item major pentatonic (1 - 2 - 3 - 5 - 6)
		\item minor pentatonic (1 - b3 - 4 - 5 - b7)
		\item blues (1 - b3 - 4 - b5 - 5 - b7)
	\end{itemize}

why scales matter:
	\begin{itemize}
		\item scales define the notes available for melody and harmony
		\item chords are built from notes within a scale
		\item modes are variabtions of scales starting on a different scale degrees
		\item understanding scales helps in improvisation, composition, and ear training
	\end{itemize}

"dia" = "through" or "across"\\
"tonic" = "tone" (referring to whole steps)\\

Diatonic literally means "through the tones" - referring to the pattern of whole tones (whole steps) and semitones (half steps) that make up the scale.\\

This pattern naturally contains the perfect 5th and the major/minor 3rd intervals, which are crucial for harmony and tonality.\\

A key signature tells you which notes are sharpened (\#) or flattened (b) consistently throughout a piece. It is written at the beginning of a staff in sheet music. It defines the key or tonal center, i.e., which scale the music is based on.\\

A mode is a type of scale derived by starting on different degrees of a parent scale.\\

A chord is a group of notes played simultaneously, usually built by stacking intervals like thirds. The simplest and most common chords are triads, made from three notes: root, third, and fifth.

\section*{circle of fifths}

The circle of fifths is a foundational concept in music theory that visually represents the relationship amoung the 12 tones of the chromatic scale, their corresponging key signatures,and the major and minor keys.\\

helps musicians understand:
	\begin{itemize}
		\item key signatures (how many sharps of flats are in a key)
		\item the relationship between major and minor keys
		\item modulation (changing keys)
		\item chord progressions
	\end{itemize}

\textbf{C, G, D, A, E, B, Gb/F\#, Db, Ab, Eb, Bb, F}\\
moving clockwise, each step goes up a perfect fifth (or down a perfect fourth), increasing in sharps and decreasing in flat (if coming from a flat key)

\end{document}
